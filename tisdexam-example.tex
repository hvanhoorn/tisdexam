\documentclass[12pt,dutch,addpoints]{tisdexam}

%% Exam credentials
% Standard TIS
%\faculteit{TIS}
\opsteller{Jesse op den Brouw}
\tweedelezer{Ad van den Bergh}
\opleiding{Elektro}
\toetsnaam{DIGTEC}
\toetsnaamkort{DIGTEC}
\toetscode{E-DIGTEC-th1}

%% Checkboxes
\voltijdtrue
\deeltijdtrue
\duaaltrue

%% More exam credentials
\toetsdatum{12 mei 2024}
\toetsdatumkort{12-05-2024}
\begintijd{8:45}
\eindtijd{9:30}
\tijdsduur{45}

%% Automatic calculated if not used
%\aantalpaginas{}
%\aantalopenvragen{}
%\aantalgeslotenvragen{}

%% Even more credentials
\cesuur{45}
\puntenverdeling{1 punt voor elke vraag}

%% More checkboxes
\toetspapiertrue
\antwoordenbladentrue
\antwoordenbladenabcdetrue
\ruitjespapiertrue
\kladpapiertrue
\tekenbenodigdhedentrue
\eenvoudigerekenmachinetrue
\grafischerekenmachinetrue
\wetenschappelijkerekenmachinetrue
\formulebladentrue
\wetbundeltrue
\aantekeningentrue
\boektrue
\overigtrue
\geenhulpmiddelentrue

%% For use with \overigtrue
\overig{Tekst bij overig}

%% Comments/remarks
\opmerkingen{Je kunt in dit vakje opmerkingen plaatsen. Als het goed is wordt alles automatisch getypeset.}

\usepackage[colorlinks,linkcolor=blue]{hyperref}

\begin{document}
\makecoverpage

\newgeometry{left=2.5cm,right=2.5cm,top=2.5cm,bottom=3cm}

\textbf{Lees vooraf:}
\begin{itemize}
\item Er zijn in totaal \numquestions\ vragen met in totaal \numparts\ deelvragen.
\item In totaal zijn \numpoints\ punten te behalen.
\item Als je een vraag niet snapt, geef dan op papier aan hoe je de vraag interpreteert.
\item Nog meer opmerkingen.
\end{itemize}

\hrulefill

\begin{questions}


\question[5]
Bereken: \[\int \mathrm{e}^{-x}\cos 2x \, \mathrm{d}x\]

\question[5]
Bereken de afgeleide van $x^2+2x+5$.

\question[2]
\label{opg:opg2}
Wat is het antwoord op de vraag der vragen?
\begin{choices}
\choice 1
\CorrectChoice \label{ans:ans2} 42
\choice 6
\choice 3
\end{choices}

\question
Dit is een inleidend stukje op de subvragen.
\begin{parts}
\part[4] Dit is een vraag. {} \droppoints
\part[5] Dit is een vraag. {} \droppoints
\part[6] Dit is een vraag. {} \droppoints
\part[7] Dit is een vraag. {} \droppoints
\end{parts}

\end{questions}

\hrulefill

Het antwoord van opgave \ref{opg:opg2} is \ref{ans:ans2}.
\end{document}
