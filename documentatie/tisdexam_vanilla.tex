%%
%%
%%   tisdexam_vanilla.tex
%%
%%   (c)2017, J.E.J. op den Brouw <J.E.J.opdenBrouw@hhs.nl>
%%
%%   v2.0 -- 2017/08/24
%%
%%   
%%


%% The document class is TISD Exam...
\documentclass[dutch,a4paper,12pt,addpoints,fleqn,concept,dyslect,vanilla]{tisdexam}


%% The info for the cover sheet needs to be set up before the beginning of document
\locatie{Delft}
\opleiding{Elektrotechniek}
\toetsnaam{INLDIG}
\toetsnaamkort{INLDIG}
\groep{EP11, EP12}
\toetsdatum{30 januari 2014}
	\toetsdatumkort{30-01-2014}
\tijd{13:00 -- 14:30 uur}
\opmerkingen{Bij dit tentamen mogen de boeken en dictaten gebruikt worden.}
\cesuur{Bij elke opgave staat het maximum aantal te behalen punten genoteerd, in totaal is maximaal \numpoints{} punten te behalen.}
\opsteller{J.E.J. op den Brouw}
\tweedelezer{J.Z.M. Broeders}
\cursuscode{E-INLDIG-th1}
\gelinieerdpapiertrue
\kladpapiertrue
\grafischerekenmachinetrue
\boekendictatentrue
\boekendictaten{zie opmerkingen}


\usepackage[T1]{fontenc}
\usepackage[latin1]{inputenc}

%% Use packages...
\usepackage{array}
\setlength{\mathindent}{1em}

%% Include graphics files
\usepackage{graphicx}

%% Enumerate items
\usepackage{enumitem}

%% Using floats
\usepackage{float}

%% Used dashed lines in tables
\usepackage{arydshln}

%% Use the AMS Mathematical characters
\usepackage{amsmath}
\usepackage{amsfonts}
\usepackage{amssymb}

%% Making captions nicer...
\usepackage[font=footnotesize,format=plain,labelfont=bf,up,textfont=it,up]{caption}
\captionsetup[table]{justification=raggedright,singlelinecheck=off,skip=4pt}

%% Command \overbar zet de bar iets hoger dan \overline
%% Werkt helaas niet genest
\newcommand*{\overbar}[1]{%
  \overline{\mbox{#1}\raisebox{2.3mm}{}}%
}
\newcommand*{\oline}[1]{\overline{#1\mathstrut}}

%% PDF Version and compression...
\pdfminorversion=5
\pdfobjcompresslevel=2

%
\usepackage{parskip}


%% At last, the document...
\begin{document}

\makecoverpage

\ifvanilla
Dit document is met de \verb|tisdexam| class opgemaakt met de optie
\verb|vanilla|. Dat houdt in dat bijna alle toegevoegde features van
de \verb|tisdexam| class zijn gedeactiveerd, alleen de
\verb|\makecoverpage| macro werkt nog.
\else
Dit document is opgemaakt in normal-mode of non-vanilla-mode. Om de
vanilla-mode te gebruiken moet de optie \verb|vanilla| in de optielijst
van de documentclass gezet worden.
\fi

\ifconcept
De \verb|concept|-optie staat aan.
\else
De \verb|concept|-optie staat uit.
\fi

\ifdyslect
De \verb|dyslect|-optie staat aan.
\else
De \verb|dyslect|-optie staat uit.
\fi

\bigskip
\bigskip

\begin{questions}
        
\question[10]
Hoeveel is $1+1$?

\question
Gegeven de functies: $f(x) = x^3+3x$ en $g(x) = x^2 +6x$
\begin{parts}
\part[3] Bepaal de nulpunten van deze functies.
\part[5] Bepaal de extremen van deze functies.
\part[7] Bepaal de snijpunten van $f(x)$ met $g(x)$.
\end{parts}

\question[5]
\label{opg:opg3}
Een gebruiker wil van R25 bit 6 en 1 inverteren en bit 4 en 2 op nul zetten.
Hiervoor zijn een EXOR- en een AND-masker nodig. Welke waardes zijn correct?
\begin{choices}
	\choice EXOR = 0xBD, AND = 0x14
	\choice EXOR = 0x42, AND = 0x14
	\choice EXOR = 0xBD, AND = 0xEB
	\CorrectChoice \label{ans:opg3} EXOR = 0x42, AND = 0xEB
\end{choices}

Bij vraag \ref{opg:opg3} is het label \verb|opg:opg3| geplaatst zodat hieraan
gerefereerd kan worden. Bij het goede antwoord is het label \verb|ans:opg3|
geplaatst. Hiermee kan het goede antwoord afgedrukt worden.

Het goede antwoord op vraag \ref{opg:opg3} is \ref{ans:opg3}.

\end{questions}


\end{document}
