%%
%%
%%   tisdexam.tex - documentation for the tisdexam document class
%%
%%   (c)2019, J.E.J. op den Brouw <J.E.J.opdenBrouw@hhs.nl>
%%
%%   v2.0e -- 2019/12/12
%%
%%   
%%


%% The document class is TISD Exam...
%\documentclass[12pt,dutch,dyslect,answers,addpoints,fleqn]{tisdexam}
%\documentclass[12pt,dutch,answers,addpoints,fleqn]{tisdexam}
%\documentclass[dutch,a4paper,12pt,answers,addpoints,fleqn]{tisdexam}
\documentclass[dutch,a4paper,12pt,addpoints,fleqn,oneside]{tisdexam}
%\documentclass[12pt,dutch,addpoints,fleqn]{tisdexam}

%% Don't create the hyperref meuk...
\makeatletter
\tisdexam@hyperrefsetupistrue
\makeatother

%% The info for the cover sheet needs to be set up before the beginning of document
\opleiding{Elektrotechniek}
\toetsnaam{INLDIG}
\toetsnaamkort{INLDIG}
\groep{EP11, EP12}
\toetsdatum{donderdag 30 januari 2014}
	\toetsdatumkort{30-01-2014}
\tijd{13:00 -- 14:30 uur}
\opmerkingen{Bij elke opgave staat het maximum aantal te behalen punten genoteerd, in totaal is maximaal \numpoints{} punten te behalen.}
\cesuur{cijfer = (aantal behaalde punten/10) + 1}
\opsteller{J.E.J. op den Brouw}
\tweedelezer{J.Z.M. Broeders}
\cursuscode{E-INLDIG-th1}
\locatie{Delft}

\gelinieerdpapiertrue
\kladpapiertrue
\grafischerekenmachinetrue
\formulebladentrue
\formulebladen{Boekje Aad Sluyter}

\documentengesorteerddocumenttrue





\usepackage[T1]{fontenc}
\usepackage[latin1]{inputenc}

%\usepackage{showframe}

%% Use packages...
\usepackage{array}

%% Include graphics files
\usepackage{graphicx}

%% Enumerate items
%\usepackage{enumitem}

%% Using floats
\usepackage{float}

%% Used dashed lines in tables
\usepackage{arydshln}

%% Use the AMS Mathematical characters
\usepackage{mathtools}
\usepackage{amsfonts}
\usepackage{amssymb}
\setlength{\mathindent}{1em}

%% Making captions nicer...
\usepackage[font=footnotesize,format=plain,labelfont=bf,textfont=sl]{caption}
\captionsetup[table]{justification=raggedright,singlelinecheck=off,skip=4pt}

%% Zet \bar altijd op een strut
\newcommand*{\oline}[1]{\overline{#1\mathstrut}}

%% English spelling of chapter, section, etc.
%\usepackage[dutch]{babel}

% Use hyperrefs in PDFs
\usepackage{hyperref}
\hypersetup{
  colorlinks=true,
  linkcolor=blue,
  pdftitle={TISD Exam Document Class},
  pdfauthor={J.E.J op den Brouw},
  pdfsubject={Toetsen maken in LaTeX met de TISD Exam Document Class},
  pdfkeywords={Latex, Document Class, TISD, Toets}
}
\definecolor{red}{rgb}{1,0,0}
\definecolor{blue}{rgb}{0,0,1}
\definecolor{darkgreen}{rgb}{0,0.4,0}
\definecolor{orange}{rgb}{1,0.5,0}
\definecolor{gray}{rgb}{0.5,0.5,0.5}
\definecolor{lightgray}{rgb}{0.95,0.95,0.95}

%% Set up the LaTex language layout
\usepackage{listings}
\lstset{ %
  language=[LaTeX]tex,
  basicstyle=\footnotesize\fontfamily{pcr}\selectfont,
  commentstyle=\itshape,
  numbers=none,
  numberstyle=\tiny\color{gray},
  stepnumber=1,
  numbersep=5pt,
  backgroundcolor=\color{lightgray},
  showspaces=false,
  showstringspaces=false,
  showtabs=false,
  frame=lines,
  rulecolor=\color{black},
  tabsize=4,
  captionpos=b,
  breaklines=true,
  breakatwhitespace=false,
  title=\lstname,
  moretexcs={setlength},
  aboveskip=15pt,
  belowskip=-10pt
}

%% Need one more footnote...
%% Display package name boldface
\renewcommand*{\thefootnote}{\fnsymbol{footnote}}
\newcommand\Package[1]{\texttt{#1}}
\newcommand\DocClass[1]{\texttt{#1}}
\newcommand\Option[1]{\texttt{#1}}

%% No indent
\newlength\myparindent
\setlength{\myparindent}{\parindent}
\setlength{\parindent}{0pt}
%% Don't use parskip
%\usepackage{parskip}

\pagestyle{headandfoot}
\runningfooter{}{\thepage}{}


%% At last, the document...
\begin{document}
\raggedbottom
\title{De \DocClass{tisdexam} document class}
\author{Jesse op den Brouw\thanks{De Haagse Hogeschool,
        \url{J.E.J.opdenBrouw@hhs.nl}}}
\date{\tisdexamfiledate, versie \tisdexamfileversion}
\maketitle
\setcounter{footnote}{1}


\section{Inleiding}
\label{sec:abstract}
De \DocClass{tisdexam} document class kan gebruikt worden om toetsen voor
de Faculteit TIS op te maken volgens een vast format. De \DocClass{tisdexam}
kan gebruikt worden met \LaTeX, lua\LaTeX\ en Xe\LaTeX.

\medskip
De class heeft een aantal opties en een aantal \textsl{commando's} (ook wel
\textsl{macro's} genoemd).

\medskip
Een belangrijk onderdeel van een toets is het verplichte voorblad. Hierop kan
allerlei informatie gegeven aan de student en surveillant. De opmaak van het
voorblad is heel eenvoudig te realiseren m.b.v.\@ commando's.

\medskip
De \DocClass{tisdexam} class is gebouwd bovenop de \DocClass{exam}
class. Alle opties uit die class zijn beschikbaar via de
\DocClass{tisdexam} class. De \DocClass{exam} class roept zijn
beurt weer de \DocClass{article} class aan.
Een aantal \DocClass{exam} class commands is gewijzigd en twee
\textsl{environments} zijn verwijderd. Zie verderop in dit document.


\section{Gebruik en class opties}
De document class wordt op de gebruikelijke manier geladen met:
\bigskip

\verb|\documentclass[| \textsl{opties} \verb|]{tisdexam}|
\bigskip

\noindent
waarbij \textsl{opties} een lijst van opties is. De opties van \DocClass{tisdexam} zijn
onderstaand opgesomd:
\smallskip

\Option{dutch} -- gebruik nederlandse spelling en toetsvoorblad

\Option{nederlands} -- gebruik nederlandse spelling  en toetsvoorblad

\Option{english} -- gebruik engelse spelling en toetsvoorblad

\Option{american} -- gebruik amerikaanse spelling en toetsvoorblad

\Option{dyslect} -- typeer het document als dyslect

\Option{concept} -- typeer het document als concept

\Option{vanilla} -- typeer het document als 'vanilla'

\Option{mimicwordtwentyten} -- bootst een uitvoer na alsof deze uit Word 2010 komt

\Option{10pt} -- geef een waarschuwing en gebruik 10 pt lettergrootte% en gebruik 11 pt lettergrootte

\Option{11pt} -- gebruik 11 pt lettergrootte

\Option{12pt} -- gebruik 12 pt lettergrootte

\bigskip
Als geen taal is opgegeven, wordt \Option{dutch} geselecteerd.
De standaard lettergrootte is 12 pt als geen lettergrootte wordt opgegeven.
De lettergrootte 10 pt wordt niet aanbevolen. De letters zijn te klein om
goed te kunnen lezen.
Opties die niet in bovenstaande lijst vermeld zijn, worden doorgegeven aan
de onderliggende \DocClass{exam} class.

\medskip
De optie \Option{dyslect} zet de boolean \Option{dyslect} op true. De boolean
kan gebruikt worden in conditionele compilatie:

\begin{lstlisting}
\ifdyslect
   \usepackage[scaled=0.92]{helvet}
   \renewcommand*\familydefault{\sfdefault}
   \usepackage[EULERGREEK]{sansmath}
   \sansmath
   \let\textit\textbf
   \let\texttt\textbf
   \let\textsl\textbf
\else
   \usepackage{mathptmx}
\fi
\end{lstlisting}
%\bigskip

%%Merk op dat het gebruik van de optie \Option{dyslect} in ieder geval er voor
%%zorgt dat een aantal stijlen (italics, slanted en teletype) veranderd wordt in
%%boldface.

\medskip
De optie \Option{vanilla} zorgt ervoor dat de \DocClass{tisdexam} class `kaal'
wordt opgeleverd. Zie hoofdstuk~\ref{sec:vanilla}.
%In feite is het nu identiek aan de \DocClass{exam} class met
%de mogelijkheid tot het genereren van voorblad.

\medskip
De optie \Option{concept} typeert het document als concept. In de rechter header
wordt het woord CONCEPT afgedrukt en op het voorblad wordt de toetscode
gevolgd door het woord (CONCEPT) (inclusief de haakjes).

\medskip
De optie \Option{mimicwordtwentyten} probeert de resulterende uitvoer zoveel
mogelijk op uitvoer van Word 2010 te laten lijken. Op dit moment wordt in de
eigenschappen van de uitgevoerde PDF de creator en producer op Word 2010 gezet.
%Het \lstinline|\TeX|-commando op het voorblad wordt niet afdrukt.

\bigskip
Een voorbeeld van het gebruik van de \DocClass{tisdexam} class:
\bigskip

\verb|\documentclass[a4paper,dutch,addpoints,fleqn,11pt]{tisdexam}|
\bigskip

Nog een voorbeeld van het gebruik van de \DocClass{tisdexam} class:
\bigskip

\verb|\documentclass[a4paper,english,addpoints,12pt,concept,dyslect]{tisdexam}|
\bigskip

De Engelse en Amerikaanse spelwijzen zijn volledig
ge\"{\i}mplementeerd. Als toetsvoorblad wordt de Engelse variant gebruikt.


\section{Commando's toetsvoorblad}
\label{sec:commands}

De \DocClass{tisdexam} class kent een groot aantal commando's waarvan de meeste
betrekking hebben op het voorblad.  Het voorblad kan gezien worden als onderverdeeld
in drie secties die elkaar niet logischerwijze opvolgen. De eerste sectie geeft de
algemene informatie over de toets, zoals
toetsnaam en opsteller. De commando's zijn hieronder te vinden.
\bigskip

\begin{tabular}{lp{11.9cm}}
\verb|\locatie| & De locatie van de toets; Delft of Den Haag (The Hague) \\
\verb|\opleiding| & De opleiding, volledig uitgeschreven\\ 
\verb|\toetsnaam| & De afkorting van de toets/vak \\
\verb|\toetsnaamkort| & De (korte) code van de toets/vak, wordt in de footer afgedrukt \\ 
\verb|\opsteller| & De opsteller van deze toets, meestal de docent die het betreffende vak geeft. \\
\verb|\tweedelezer| & De tweede lezer, een docent die de toets gecontroleerd heeft op correctheid. \\
\verb|\toetsdatum| & De datum waarop de toets gegeven wordt, met de maandnaam volledig uitgeschreven \\ 
\verb|\toetsdatumkort| & Korte notatie van de toetsdatum in dd-mm-jjjj \\
\verb|\tijd|       & De tijdvak waarin de toets gegeven wordt, begin- en eindtijd \\
\verb|\groep|     & De groep of groepen waarvoor de toets bedoeld is \\ 
\verb|\cursuscode|    & De cursuscode van het vak. \\
\verb|\aantalbladzijden| & Het aantal pagina's waaruit de toets bestaat, inclusief het voorblad \\
\verb|\aantalvragen| & Het aantal opgaven waaruit de toets bestaat \\
\verb|\opmerkingen| & Een vrij veld waar allerlei opmerkingen kunnen worden geplaatst \\
\verb|\cesuur| & Een vrij veld waarin de cesuur van de toets gepresenteerd wordt \\
\end{tabular}

\bigskip
De tweede sectie geeft aan wat er allemaal bij de toets wordt verstrekt en welke hulpmiddelen gebruikt
mogen worden. De commando's zijn hieronder te vinden. In feite zijn het allemaal booleans die op true
of false (default) gezet kunnen worden.

\bigskip
\begin{tabular}{lp{8cm}}
\verb|\gelinieerdpapiertrue| & Gelinieerd papier \\
\verb|\opgavenbladentrue| & Opgavebladen met ruimte om de vragen te beantwoorden \\
\verb|\ruitjespapiertrue| & Ruitjespapier \\
\verb|\kladpapiertrue| & Kladpapier \\
\verb|\antwoordformulierabcdetrue| & Antwoordformulier ABCDE \\
\verb|\omslaggemaakttentamentrue| & Gelinieerd omslag voor het tentamen \\
\verb|\antwoordformulierjaneetrue| & Antwoordformulier Ja/Nee \\
\verb|\overigtrue| & Overige zaken die verstrekt worden. Zie ook \verb|\overig| \\
\verb|\antwoordformulierjaneevraagtekentrue| & Antwoordformulier Ja/Nee/Vraagteken \\
\verb|\bijlagentrue| & Bijlagen die verstrekt worden. Zie ook \verb|\bijlagen| \\
\verb|\eenvoudigerekenmachinetrue| & Gebruik eenvoudige rekenmachine \\ 
\verb|\tekenbenodigdhedentrue| & Gebruik van tekenbenodigdheden (passer, liniaal) \\
\verb|\grafischerekenmachinetrue| & Gebruik grafische rekenmachine \\
\verb|\eigenaantekeningentrue| & Gebruik eigen aantekeningen. Zie ook \verb|\eigenaantekeningen| \\
\verb|\computertrue| & Gebruik computer \\
\verb|\boekendictatentrue| & Gebruik van boeken en dictaten. Zie ook \verb|\boekendictaten| \\
\verb|\formulebladentrue| & Gebruik formulebladen. Zie ook \verb|\formulebladen| \\
\end{tabular}

\medskip
Bij deze booleans zijn nog een paar velden mogelijk:

\bigskip
\begin{tabular}{lp{15cm}}
\verb|\overig| & Wat verder verstrekt wordt \\
\verb|\bijlagen| & De bijlagen \\
\verb|\formulebladen| & Welke formulebladen gebruikt mag worden \\
\verb|\eigenaantekeningen| & Eigen aantekeningen \\
\verb|\boekendictaten| & Welke boeken en dictaten gebruikt mogen worden \\
\end{tabular}

\bigskip
Deze velden hebben een beperkte ruimte. Verwijs liever naar de opmerkingen en vul bij \verb|\opmerkingen|
de benodigde informatie in.

\bigskip
Er zijn nog twee booleans die gezet kunnen worden. Slechts \'e\'en van de twee mag gezet worden.

\bigskip
\begin{tabular}{lp{9cm}}
\verb|\documentengesorteerddocumenttrue| & Alle documenten voorzien van naam en studentnummer, per document gesorteerd \\
\verb|\documentengesorteerdstudenttrue| & Alle documenten voorzien van naam en studentnummer, per student gesorteerd (in omslag) \\
\end{tabular}

%%%\bigskip
%%%De argumenten bij de commando's \verb|\toetsnaamkort|, \verb|\groep|, \verb|\toetsdatum|
%%%en \verb|\toetsinleveren| worden volledig met hoofdletters afgedrukt.

\medskip
Het argument van het commando \verb|\toetsdatumkort| wordt niet op het voorblad
geplaatst, maar in de footer van de vervolgbladen.

\medskip
Het aantal pagina's van de toets is op te vragen door het commando \verb|\numpages| en
kan direct via het commando \verb|\aantalbladzijden| worden ingesteld (dit is bij default
al gedaan):

\bigskip
\verb|\aantalbladzijden{\numpages}|

\bigskip
Het aantal toetsvragen van de toets is op te vragen door het commando's \verb|\numopenquestions| en
\verb|\nummultquestions| voor respectievelijk open vragen en multiple choice vragen.
Het commando \verb|\aantalvragen| kan hiermee worden ingesteld:

\bigskip
\verb|\aantalvragen{\numopenquestions{} open vragen, \nummultquestions{}|

\verb|                                                       meerkeuzevragen}|

Default is het totaal aantal vragen ingesteld (\verb|\numquestions|).

%%%\bigskip
%%%Bij het commando \verb|\gebruikrekenmachine| moet \'e\'en van de volgende opties worden
%%%ingevuld: Geen rekenmachine toegestaan, Alleen gewone rekenmachine, Grafische rekenmachine
%%%(let hierbij op de hoofdletter aan het begin). De standaard waarde is Grafische rekenmachine.

\bigskip
Het maximaal aantal te behalen punten voor de hele toets is op te vragen via het commando
\verb|\numpoints| en kan bijvoorbeeld als invoer dienen voor het commando \verb|\opmerkingen| of
\verb|\cesuur|:

\bigskip
\verb|\opmerkingen{Maximaal aantal te behalen punten is \numpoints.}|

%%%\bigskip
%%%NOOT: het commando \verb|\hulpmiddelen| is verwijderd vanaf versie 1.7.

\bigskip
Het commando \verb|\makecoverpage| genereert het verplichte toetsvoorblad.
Een voorbeeld is te vinden op pagina~\pageref{pag:coverpage}. Bij het vormgeven
van het toetsvoorblad zijn onderstaande commando's gebruikt.

\newpage
\begin{lstlisting}
\opleiding{Elektrotechniek}
\toetsnaam{INLDIG}
\toetsnaamkort{INLDIG}
\groep{EP11, EP12}
\toetsdatum{donderdag 30 januari 2014}
	\toetsdatumkort{30-01-2014}
\tijd{13:00 -- 14:30 uur}
\opmerkingen{Bij elke opgave staat het maximum aantal te behalen punten genoteerd, in totaal is maximaal \numpoints{} punten te behalen.}
\cesuur{cijfer = (aantal behaalde punten/10) + 1}
\opsteller{J.E.J. op den Brouw}
\tweedelezer{J.Z.M. Broeders}
\cursuscode{E-INLDIG-th1}
\locatie{Delft}

\gelinieerdpapiertrue
\kladpapiertrue
\grafischerekenmachinetrue
\formulebladentrue
\formulebladen{Boekje Aad Sluyter}

\documentengesorteerddocumenttrue
\end{lstlisting}

\section{Overige commando's}
\label{sec:overigecom}

De \DocClass{tisdexam} class laadt automatisch de \Package{hyperref} package
en stelt enkele parameters m.b.t.\@ het PDF-bestand in, zoals de titel.

%\medskip\noindent
Het aanroepen van de \Package{hyperref} package gebeurt pas op het allerlaatste
moment, vlak voor het begin van het document. Er is een aantal packages die
pas na de \Package{hyperref} package moeten worden geladen zoals de \Package{algorithm}
package.

%\medskip\noindent
Het commando \verb|\setuphyperrefheaderfooter| zorgt ervoor dat \Package{hyperref}
wordt uitgevoerd op het moment van uitvoeren van het commando. Voorbeeld:

\begin{lstlisting}
  ...
\usepackage{float}
\setuphyperrefheaderfooter % hyperref package loaded
\usepackage{algorithm}

\begin{document}
  ...
\end{lstlisting}

\medskip\noindent
Met het command \verb|\nosetuphyperrefheaderfooter| wordt het automatisch
gebruik van de \Package{hyperref} package \'{e}n headers en footers uitgezet.
De gebruiker moet dan zelf de package laden en opzetten. Dat geldt ook
voor het gebruik van headers en footers.

%\medskip\noindent
%Het commando \verb|\numopenquestions| expandeert tot het aantal open vragen.

%\medskip\noindent
%Het commando \verb|\nummultquestions| expandeert tot het aantal multiple choice vragen.

\section{Aangepaste \DocClass{exam} class commando's}
Een aantal commando's uit de \DocClass{exam} class is aangepast.
\medskip

\verb|\pointname| -- opnieuw gedefinieerd als \verb|{ pt}| (let op de spatie).

\verb|\marginpointname| -- opnieuw gedefinieerd als \verb|{ pt}| (let op de spatie).

\verb|\partlabel| -- opnieuw gedefinieerd als \verb|\thepartno)| (let op het haakje).

\verb|\thechoice| -- opnieuw gedefinieerd als \verb|\alph{choice}|.

\verb|\choicelabel| -- opnieuw gedefinieerd als \verb|\thechoice)| (let op het haakje).

\medskip\noindent
Het commando \verb|\droppoints| is opnieuw gedefinieerd als

\begin{lstlisting}
\def\droppoints{%
  \leavevmode\unskip\nobreak
  {\padded@point@block}%
  %\par
}
\end{lstlisting}

\medskip\noindent
Het commando \verb|\qformat| is opnieuw gedefinieerd als

\begin{lstlisting}
\qformat{\textbf{\tisdexam@opgavespel{} \thequestion{}} (\totalpoints
         \@pointname) \hfill}
\end{lstlisting}


\section{Verwijderde environments}
De \DocClass{exam} class kent de environments \Package{subpart} en
\Package{subsubpart}. Deze zijn verwijderd uit de \DocClass{tisdexam} class.
Ten eerste zijn opgaven met meer dan \'{e}\'{e}n niveau didactisch niet
echt verantwoord en ten tweede is het niet meer mogelijk om het aantal
open vragen en meerkeuzevragen te tellen omdat er mengvormen mogelijk zijn.


\section{Vanilla}
\label{sec:vanilla}
De \DocClass{tisdexam} class dwingt de gebruiker het document op te stellen
in een bepaald format. Niet iedere gebruiker wil dat. De optie
\Option{vanilla} zorgt ervoor dat het document wordt opgemaakt in de vanilla-%
modus. In feite is het nu identiek aan de \DocClass{exam} class met de
mogelijkheid tot het genereren van voorblad, alle andere instellingen blijven
in de originele staat. De gebruiker moet dan zelf alle opmaak instellen
zoals headers, footers, paginanummering en hyperlinks.

De taalopties worden wel herkend en de \Package{babel} package wordt geladen
met de geselecteerde taal.

\section{Gebruikte packages}
In de `non-vanilla mode' worden de volgende packages expliciet geladen:

\texttt{xcolor}, \texttt{babel}, \texttt{geometry}, \texttt{hyperref}, \texttt{amssymb}

\medskip
In de `vanilla-mode' worden de volgende packages expliciet geladen:

\texttt{xcolor}, \texttt{babel}, \texttt{geometry}, \texttt{amssymb}


\section{Meldingen}
Een aantal meldingen kunnen gegenereerd worden, zie hieronder.

\begin{description}\itemsep-3pt
\item[\texttt{Do not use font size `10pt'. It is too small to read.}]
  \hfill \\ Gebruiker heeft de 10pt-optie gebruikt. Normale tekst wordt wel
            in 10 pt afgedrukt.
\item[\texttt{Document is typed dyslect}]
   \hfill \\ De optie \verb|dyslect| is geactiveerd.
\item[\texttt{Document is typed vanilla}]
  \hfill \\  De optie \verb|vanilla| is geactiveerd.
\item[\texttt{Document is typed concept}]
   \hfill \\ De optie \verb|concept| is geactiveerd.
\item[\texttt{No language specified, set to 'dutch'.}]
   \hfill \\ Er is geen taal geselecteerd, Nederlands is de standaardinstelling.  
\item[\texttt{Subparts not allowed in this class}]
   \hfill \\ Er kunnen geen subparts gebruikt worden in deze class.
\item[\texttt{Subsubparts not allowed in this class}]
  \hfill \\ Er kunnen geen subsubparts gebruikt worden in deze class.
\item[\texttt{Document is rendered with answers}]
 \hfill \\ De optie \verb|answers| is gebruikt, de antwoorden/uitwerkingen
           worden afgedrukt.
\item[\texttt{Package hyperref and header and footers automaticly set up at begin of document}]
  De pagina-opmaak en linking wordt automatisch geladen.
\item[\texttt{Package parskip detected. Please view the coverpage for correct rendering}]
  De package \verb|parskip| is geladen. Let er op dat \verb|parskip|
            een aantal aanpassingen doet aan layout-parameters, bv\@ spati\"{e}ring bij lijsten.
\item[\texttt{Document option mimicwordtwentyten is in effect}]
 \hfill \\ De optie \verb|mimicwordtwentyten| is gebruikt.
\end{description}



\section{Opgelet!}
De volgende commando's uit de \DocClass{exam} class mogen niet gebruikt
worden: \verb|\checkboxeshook|, \verb|\choiceshook| en \verb|\questionshook|.
Deze worden door de \DocClass{tisdexam} class gebruikt voor interne doeleinden.
Gebruik de nieuwe commando's \verb|\tisdexamcheckboxeshook|, \\
\verb|\tisdexamchoiceshook| en \verb|\tisdexamquestionshook|. Uiteraard geldt
dit niet als de vanilla-modus wordt gebruikt.

\medskip
De package \verb|geometry| wordt geladen met een standaard instelling. Als
gebruiker kan deze package dus niet geladen worden. Gebruik het commando
\verb|newgeometry| om de pagina-layout in te stellen. Zie ook volgende
hoofdstuk.

%In de resulterende PDF-uitvoer is een \textsl{custom} eigenschap opgenomen
%(zie Document Properties, tabblad Custom) met de naam PTEX.fullbanner.
%Deze kan alleen met de hand verwijderd worden.

\section{Tips}
Standaard is de linkkleur op blauw gezet. Een andere kleur kan eenvoudig
geselecteerd worden door het volgende net onder het begin van het document
te zetten. De links worden in kleur afgedrukt, dus niet met een omranding.

\begin{lstlisting}
\begin{document}
\hypersetup{linkcolor=red}
  ...
\end{lstlisting}

\medskip
Op een (papieren) toets zijn kleuren misschien niet handig, maar in een
digitale versie met antwoorden weer wel:

\begin{lstlisting}
%% Set link color to black for exam text only.
\ifprintanswers\else\hypersetup{linkcolor=black}\fi
\end{lstlisting}

\medskip
Standaard is de witruimte tussen het eind van een vraag en het begin van een
nieuwe vraag ingesteld op 0,7 cm. Om deze instelling te wijzigen moet de
lengte \verb|\questionsep| aangepast worden.

\begin{lstlisting}
\setlength{\questionsep}{0.4cm}
\end{lstlisting}

\medskip
De \verb|geometry| package wordt geladen en de linker en rechter marge worden
ingesteld op 2~cm. De pagina-layout kan opnieuw worden ingesteld
d.m.v.\@ het commando \verb|\newgeometry|.

\begin{lstlisting}
\newgeometry{bindingoffset=0.2in,left=1.0in,right=1.0in,top=0.6in,
             bottom=0.4in,footskip=.2in}
\end{lstlisting}

\section{Voorbeeldvragen}
In dit hoofdstuk wordt een aantal voorbeeldvragen gegeven

\medskip\noindent

\begin{questions}
Voorbeeld van een enkele open vraag:
         
\begin{lstlisting}
\question[10]
Hoeveel is $1+1$?
\end{lstlisting}
\question[10]
Hoeveel is $1+1$?


\bigskip\noindent
Voorbeeld van een open vraag met subvragen:

\begin{lstlisting}
\question
Gegeven de functies: $f(x) = x^3+3x$ en $g(x) = x^2 + 6x$
\begin{parts}
\part[3] \droppoints\ Bepaal de nulpunten van deze functies.
\part[5] \droppoints\ Bepaal de extremen van deze functies.
\part[7] \droppoints\ Bepaal de snijpunten van $f(x)$ met $g(x)$.
\end{parts}
\end{lstlisting}
\question
Gegeven de functies: $f(x) = x^3+3x$ en $g(x) = x^2 +6x$
\begin{parts}
\part[3] \droppoints\ Bepaal de nulpunten van deze functies.
\part[5] \droppoints\ Bepaal de extremen van deze functies.
\part[7] \droppoints\ Bepaal de snijpunten van $f(x)$ met $g(x)$.
\end{parts}

\bigskip
Voorbeeld vraag uit een toets Belastingrecht van het NCOI (examennummer:
96038, dd 23 november 2013) met andere benaming voor
punten.

\begin{lstlisting}
\pointname{ punten}
\marginpointname{ punten}
\question
Ondernemer Hans Bakker (alleenstaand, 30 jaar) heeft al sinds 2003 een
eenmanszaak. Hij verricht bouwkundige inspecties bij zijn klanten. Sinds
2009 heeft hij voor 40 uren per week Klaas Klaver in vaste dienst. Uit de
administratie van Hans over het jaar 2012 blijkt dat hij 1.600 uren heeft
gewerkt. Paula Hartman, de vriendin van Hans, woont 10 kilometer bij hem
vandaan en zij komt Hans sinds 2011 elke zaterdagochtend 3 uren assisteren
met de administratie. Zij ontvangt daarvoor geen enkele vergoeding. Het
betreft een soort van vriendendienst. Behalve dat Hans voor de
inkomstenbelasting belastingplichtig is, moet hij ook voor de omzetbelasting
aangifte doen. Gezien zijn omzet moet hij elk kwartaal zijn btw-aangifte
invullen.

\begin{parts}
\part[8] Welke tabel moet Hans Bakker hanteren in verband met de loonbelasting
van personeelslid Klaas Klaver? Motiveer uw antwoord. {} \droppoints
\part[7] Heeft ondernemer Hans Bakker over 2013 recht op de
zelfstandigenaftrek als zijn situatie gelijk is aan die van 2012? Motiveer uw
antwoord. {} \droppoints
\part[6] Kan ondernemer Hans Bakker over 2013 een bedrag opvoeren in verband
met de meewerkaftrek als zijn situatie gelijk is aan die van 2012? Motiveer
uw antwoord. {} \droppoints
\part[8] Onder welk btw-tarief vallen de werkzaamheden die ondernemer Hans
Bakker verricht? Motiveer uw antwoord. {} \droppoints
\end{parts}
\end{lstlisting}

\pointname{ punten}
\marginpointname{ punten}
\question
Ondernemer Hans Bakker (alleenstaand, 30 jaar) heeft al sinds 2003 een
eenmanszaak. Hij verricht bouwkundige inspecties bij zijn klanten. Sinds
2009 heeft hij voor 40 uren per week Klaas Klaver in vaste dienst. Uit de
administratie van Hans over het jaar 2012 blijkt dat hij 1.600 uren heeft
gewerkt. Paula Hartman, de vriendin van Hans, woont 10 kilometer bij hem
vandaan en zij komt Hans sinds 2011 elke zaterdagochtend 3 uren assisteren
met de administratie. Zij ontvangt daarvoor geen enkele vergoeding. Het
betreft een soort van vriendendienst. Behalve dat Hans voor de
inkomstenbelasting belastingplichtig is, moet hij ook voor de omzetbelasting
aangifte doen. Gezien zijn omzet moet hij elk kwartaal zijn btw-aangifte
invullen.

\begin{parts}
\part[8] Welke tabel moet Hans Bakker hanteren in verband met de loonbelasting
van personeelslid Klaas Klaver? Motiveer uw antwoord. {} \droppoints
\part[7] Heeft ondernemer Hans Bakker over 2013 recht op de
zelfstandigenaftrek als zijn situatie gelijk is aan die van 2012? Motiveer uw
antwoord. {} \droppoints
\part[6] Kan ondernemer Hans Bakker over 2013 een bedrag opvoeren in verband
met de meewerkaftrek als zijn situatie gelijk is aan die van 2012? Motiveer
uw antwoord. {} \droppoints
\part[8] Onder welk btw-tarief vallen de werkzaamheden die ondernemer Hans
Bakker verricht? Motiveer uw antwoord. {} \droppoints
\end{parts}

\bigskip\noindent
Voorbeeld van een multiple choice vraag:

\begin{lstlisting}
\question[5]
\label{opg:opg3}
Een gebruiker wil van R25 bit 6 en 1 inverteren en bit 4 en 2 op nul
zetten. Hiervoor zijn een EXOR- en een AND-masker nodig. Welke waardes
zijn correct?
\begin{choices}
	\choice EXOR = 0xBD, AND = 0x14
	\choice EXOR = 0x42, AND = 0x14
	\choice EXOR = 0xBD, AND = 0xEB
	\CorrectChoice \label{ans:opg3} EXOR = 0x42, AND = 0xEB
\end{choices}
\end{lstlisting}

\pointname{ pt}
\marginpointname{ pt}
\question[5]
\label{opg:opg3}
Een gebruiker wil van R25 bit 6 en 1 inverteren en bit 4 en 2 op nul zetten.
Hiervoor zijn een EXOR- en een AND-masker nodig. Welke waardes zijn correct?
\begin{choices}
	\choice EXOR = 0xBD, AND = 0x14
	\choice EXOR = 0x42, AND = 0x14
	\choice EXOR = 0xBD, AND = 0xEB
	\CorrectChoice \label{ans:opg3} EXOR = 0x42, AND = 0xEB
\end{choices}

Bij vraag \ref{opg:opg3} is het label \lstinline|opg:opg3| geplaatst zodat hieraan
gerefereerd kan worden. Bij het goede antwoord is het label \lstinline|ans:opg3|
geplaatst. Hiermee kan het goede antwoord afgedrukt worden.

\begin{lstlisting}
Het goede antwoord op vraag \ref{opg:opg3} is \ref{ans:opg3}.
\end{lstlisting}
Het goede antwoord op vraag \ref{opg:opg3} is \ref{ans:opg3}.

\end{questions}


\newpage
\label{pag:coverpage}
\noprintanswers
%\toetsnaamkort{INLDIG (herkansing)}
%\setlength{\parindent}{\myparindent}

\thispagestyle{empty}
%\newgeometry{bindingoffset=0.0in,left=2cm,right=2cm,top=1in,bottom=1in,footskip=.40in}

\makecoverpage


\end{document}
