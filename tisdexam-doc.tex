%%
%%
%%   tisdexam-doc.tex - documentation for the tisdexam document class
%%
%%   (c)2023, J.E.J. op den Brouw <J.E.J.opdenBrouw@hhs.nl>
%%
%%   v3.0\alpha -- 2023/12/02
%%
%%   
%%


%% The document class is TISD Exam...
\documentclass[12pt,dutch,addpoints,fleqn]{tisdexam}

\newgeometry{left=2.0cm,right=2.0cm,top=2.0cm,bottom=1.4in}

%% The info for the cover sheet needs to be set up before the beginning of document
%% Exam credentials
\opsteller{Jesse op den Brouw}
\opleiding{Elektro}
\toetsnaam{DIGTEC}
\toetsnaamkort{DIGTEC}
\toetscode{E-DIGTEC-th1}

%% Checkboxes
\voltijdtrue
\deeltijdtrue
\duaaltrue

%% More exam credentials
\toetsdatum{24 november 2023}
\toetsdatumkort{24-11-2023}
\begintijd{8:45}
\eindtijd{9:30}
\tijdsduur{45}

%% Automatic calculated if not used
%\aantalpaginas{}
%\aantalopenvragen{}
%\aantalgeslotenvragen{}

%% Even mor credentials
\cesuur{45}
\puntenverdeling{1 punt voor elke vraag}

%% More checkboxes
\toetspapiertrue
\antwoordenbladentrue
\antwoordenbladenabcdetrue
\ruitjespapiertrue
\kladpapiertrue
\tekenbenodigdhedentrue
\eenvoudigerekenmachinetrue
\grafischerekenmachinetrue
\formulebladentrue
\wetbundeltrue
\aantekeningentrue
\boektrue
\overigtrue
\geenhulpmiddelentrue

%% For use with \overigtrue
\overig{Regeltje tekst}

%% Cooments/remarks
\opmerkingen{Je kunt in dit vakje opmerkingen plaatsen. Als het goed is wordt alles automatisch getypeset.}





%\usepackage{showframe}

%% Use packages...
\usepackage{array}

%% Include graphics files
\usepackage{graphicx}

%% Enumerate items
%\usepackage{enumitem}

%% Using floats
\usepackage{float}


%% Use the AMS Mathematical characters
\usepackage{mathtools}
\usepackage{amsfonts}
\usepackage{amssymb}
\setlength{\mathindent}{1em}

%% Making captions nicer...
\usepackage[font=footnotesize,format=plain,labelfont=bf,textfont=sl]{caption}
\captionsetup[table]{justification=raggedright,singlelinecheck=off,skip=4pt}

%% Zet \bar altijd op een strut
\newcommand*{\oline}[1]{\overline{#1\mathstrut}}

%% English spelling of chapter, section, etc.
%\usepackage[dutch]{babel}

% Use hyperrefs in PDFs
\definecolor{red}{rgb}{1,0,0}
\definecolor{blue}{rgb}{0,0,1}
\definecolor{darkgreen}{rgb}{0,0.4,0}
\definecolor{orange}{rgb}{1,0.5,0}
\definecolor{gray}{rgb}{0.5,0.5,0.5}
\definecolor{lightgray}{rgb}{0.95,0.95,0.95}

%% Set up the LaTex language layout
\usepackage{listings}
\lstset{ %
  language=[LaTeX]tex,
  basicstyle=\footnotesize\fontfamily{pcr}\selectfont,
  commentstyle=\itshape,
  numbers=none,
  numberstyle=\tiny\color{gray},
  stepnumber=1,
  numbersep=5pt,
  backgroundcolor=\color{lightgray},
  showspaces=false,
  showstringspaces=false,
  showtabs=false,
  frame=lines,
  rulecolor=\color{black},
  tabsize=4,
  captionpos=b,
  breaklines=true,
  breakatwhitespace=false,
  title=\lstname,
  moretexcs={setlength},
  aboveskip=15pt,
  belowskip=-10pt
}

%% Need one more footnote...
%% Display package name boldface
\renewcommand*{\thefootnote}{\fnsymbol{footnote}}
\newcommand\Package[1]{\texttt{#1}}
\newcommand\DocClass[1]{\texttt{#1}}
\newcommand\Option[1]{\texttt{#1}}

\usepackage{parskip}

\pagestyle{headandfoot}
\runningfooter{}{\thepage}{}


%% At last, the document...
\begin{document}
\raggedbottom
\title{De \DocClass{tisdexam} document class}
\author{Jesse op den Brouw\thanks{De Haagse Hogeschool,
        {J.E.J.opdenBrouw@hhs.nl}}}
\date{\tisdexamfiledate, versie \tisdexamfileversion}
\maketitle
\setcounter{footnote}{1}


\section{Inleiding}
\label{sec:abstract}
De \DocClass{tisdexam} document class kan gebruikt worden om toetsen voor
de Faculteit TIS op te maken volgens een vast format. De \DocClass{tisdexam}
kan gebruikt worden met \LaTeX, lua\LaTeX\ en Xe\LaTeX.

De class heeft een aantal opties en een aantal \textsl{commando's} (ook wel
\textsl{macro's} genoemd).

Een belangrijk onderdeel van een toets is het verplichte voorblad. Hierop kan
allerlei informatie gegeven aan de student en surveillant. De opmaak van het
voorblad is heel eenvoudig te realiseren m.b.v.\@ commando's.

De \DocClass{tisdexam} class is gebouwd bovenop de \DocClass{exam}
class. Alle opties uit die class zijn beschikbaar via de
\DocClass{tisdexam} class. De \DocClass{exam} class roept zijn
beurt weer de \DocClass{article} class aan.
Een aantal \DocClass{exam} class commands is gewijzigd en twee
\textsl{environments} zijn verwijderd. Zie verderop in dit document.


\section{Gebruik en class opties}
De document class wordt op de gebruikelijke manier geladen met:

\verb|\documentclass[| \textsl{opties} \verb|]{tisdexam}|

waarbij \textsl{opties} een lijst van opties is. De opties van \DocClass{tisdexam} zijn
onderstaand opgesomd:

\Option{dutch} -- gebruik nederlandse spelling en toetsvoorblad

\Option{nederlands} -- gebruik nederlandse spelling  en toetsvoorblad

\Option{english} -- gebruik engelse spelling en toetsvoorblad

\Option{american} -- gebruik amerikaanse spelling en toetsvoorblad

\Option{vanilla} -- typeer het document als 'vanilla'


Als geen taal is opgegeven, wordt \Option{dutch} geselecteerd.
Opties die niet in bovenstaande lijst vermeld zijn, worden doorgegeven aan
de onderliggende \DocClass{exam} class.

De optie \Option{vanilla} zorgt ervoor dat de \DocClass{tisdexam} class `kaal'
wordt opgeleverd. Zie hoofdstuk~\ref{sec:vanilla}.
%In feite is het nu identiek aan de \DocClass{exam} class met
%de mogelijkheid tot het genereren van voorblad.

Een voorbeeld van het gebruik van de \DocClass{tisdexam} class:

\verb|\documentclass[a4paper,dutch,addpoints,fleqn,11pt]{tisdexam}|

Nog een voorbeeld van het gebruik van de \DocClass{tisdexam} class:

\verb|\documentclass[a4paper,english,addpoints,12pt]{tisdexam}|

De Engelse en Amerikaanse spelwijzen zijn volledig
ge\"{\i}mplementeerd. Als toetsvoorblad wordt de Engelse variant gebruikt.


\section{Commando's toetsvoorblad}
\label{sec:commands}

De \DocClass{tisdexam} class kent een groot aantal commando's waarvan de meeste
betrekking hebben op het voorblad.  Het voorblad kan gezien worden als onderverdeeld
in drie secties die elkaar niet logischerwijze opvolgen. De eerste sectie geeft de
algemene informatie over de toets, zoals
toetsnaam en opsteller. De commando's zijn hieronder te vinden.

\begin{tabular}{lp{11.9cm}}
\verb|\faculteit| & De faculteit (standaard TIS)\\
\verb|\opleiding| & De opleiding, volledig uitgeschreven\\ 
\verb|\toetsnaam| & De afkorting van de toets/vak \\
\verb|\toetsnaamkort| & De (korte) code van de toets/vak, wordt in de footer afgedrukt \\ 
\verb|\opsteller| & De opsteller van deze toets, meestal de docent die het betreffende vak geeft. \\
\verb|\toetsdatum| & De datum waarop de toets gegeven wordt, met de maandnaam volledig uitgeschreven \\ 
\verb|\toetsdatumkort| & Korte notatie van de toetsdatum in dd-mm-jjjj \\
\verb|\begintijd|       & De tijdvak waarin de toets gegeven wordt, begin- en eindtijd \\
\verb|\eindgroep|     & De groep of groepen waarvoor de toets bedoeld is \\ 
\verb|\tijdsduur| & Tijdsduur van de toets \\
\verb|\cursuscode|    & De cursuscode van het vak. \\
\verb|\aantalpaginas| & Het aantal pagina's waaruit de toets bestaat, inclusief het voorblad \\
\verb|\aantalopenvragen| & Het aantal open opgaven waaruit de toets bestaat \\
\verb|\aantalgeslotenvragen| & Het aantal gesloten opgaven waaruit de toets bestaat \\
\verb|\opmerkingen| & Een vrij veld waar allerlei opmerkingen kunnen worden geplaatst \\
\verb|\cesuur| & Een vrij veld waarin de cesuur van de toets gepresenteerd wordt \\
\end{tabular}

De tweede sectie geeft aan wat er allemaal bij de toets wordt verstrekt en welke hulpmiddelen gebruikt
mogen worden. De commando's zijn hieronder te vinden. In feite zijn het allemaal booleans die op true
of false (default) gezet kunnen worden.

\begin{tabular}{lp{10cm}}
\verb|\voltijdtrue| & Toets voor voltijd \\
\verb|\deeltijdtrue| & toets voor deeltijd \\
\verb|\duaaltrue| & toets voor duaal \\
\verb|\toetspapiertrue| & Gelinieerd papier \\
\verb|\antwoordenbladentrue| & Opgavebladen met ruimte om de vragen te beantwoorden \\
\verb|\antwoordenbladenabcdetrue| & Antwoordformulier ABCDE \\
\verb|\ruitjespapiertrue| & Ruitjespapier \\
\verb|\kladpapiertrue| & Kladpapier \\
\verb|\tekenbenodigdhedentrue| & Gebruik van tekenbenodigdheden (passer, liniaal) \\
\verb|\eenvoudigerekenmachinetrue| & Gebruik eenvoudige rekenmachine \\ 
\verb|\grafischerekenmachinetrue| & Gebruik grafische rekenmachine \\
\verb|\formulebladentrue| & Gebruik formulebladen. \\
\verb|\wetbundeltrue| & Wetbundel. \\
\verb|\aantekeningentrue| & Gebruik eigen aantekeningen. \\
\verb|\boektrue| & Gebruik van boeken  \\
\verb|\overigtrue| & Overige zaken die verstrekt worden. Zie ook \verb|\overig| \\
\verb|\geenhulpmiddelentrue| & Geen hulpmiddelen toegestaan.
\end{tabular}


Bij deze booleans zijn nog een paar velden mogelijk:

\begin{tabular}{lp{15cm}}
\verb|\overig| & Wat verder verstrekt wordt. \\
\verb|\opmerkingen| & opmerkingen.
\end{tabular}

Het argument van het commando \verb|\toetsdatumkort| wordt niet op het voorblad
geplaatst, maar in de footer van de vervolgbladen.

Het aantal pagina's van de toets is op te vragen door het commando \verb|\numpages| en
kan direct via het commando \verb|\aantalpaginas| worden ingesteld (dit is bij default
al gedaan):

\verb|\aantalpaginas{\numpages}|

Het aantal toetsvragen van de toets is op te vragen door het commando's \verb|\numopenquestions| en
\verb|\nummultquestions| voor respectievelijk open vragen en multiple choice vragen.
De commando's \verb|\aantalopenvragen| en \verb|\aantalgeslotenvragen| kunnen hiermee worden ingesteld:

\verb|\aantalopenvragen{\numopenquestions}| \\
\verb|\aantalgeslotenvragen{\nummultquestions}|

Het maximaal aantal te behalen punten voor de hele toets is op te vragen via het commando
\verb|\numpoints| en kan bijvoorbeeld als invoer dienen voor het commando \verb|\opmerkingen| of
\verb|\cesuur|:

\verb|\opmerkingen{Maximaal aantal te behalen punten is \numpoints.}|

\newpage

\begin{lstlisting}
\opsteller{Jesse op den Brouw}
\opleiding{Elektro}
\toetsnaam{DIGTEC}
\toetsnaamkort{DIGTEC}
\toetscode{E-DIGTEC-th1}

%% Checkboxes
\voltijdtrue
\deeltijdtrue
\duaaltrue

%% More exam credentials
\toetsdatum{24 november 2023}
\toetsdatumkort{24-11-2023}
\begintijd{8:45}
\eindtijd{9:30}
\tijdsduur{45}

%% Automatic calculated if not used
%\aantalpaginas{}
%\aantalopenvragen{}
%\aantalgeslotenvragen{}

%% Even mor credentials
\cesuur{45}
\puntenverdeling{1 punt voor elke vraag}

%% More checkboxes
\toetspapiertrue
\antwoordenbladentrue
\antwoordenbladenabcdetrue
\ruitjespapiertrue
\kladpapiertrue
\tekenbenodigdhedentrue
\eenvoudigerekenmachinetrue
\grafischerekenmachinetrue
\formulebladentrue
\wetbundeltrue
\aantekeningentrue
\boektrue
\overigtrue
\geenhulpmiddelentrue

%% For use with \overigtrue
\overig{Regeltje tekst}

%% Cooments/remarks
\opmerkingen{Je kunt in dit vakje opmerkingen plaatsen. Als het goed is wordt alles automatisch getypeset.}
\end{lstlisting}


\section{Aangepaste \DocClass{exam} class commando's}
Een aantal commando's uit de \DocClass{exam} class is aangepast.

\verb|\pointname| -- opnieuw gedefinieerd als \verb|{ pt}| (let op de spatie).

\verb|\marginpointname| -- opnieuw gedefinieerd als \verb|{ pt}| (let op de spatie).

\verb|\partlabel| -- opnieuw gedefinieerd als \verb|\thepartno)| (let op het haakje).

\verb|\thechoice| -- opnieuw gedefinieerd als \verb|\alph{choice}|.

\verb|\choicelabel| -- opnieuw gedefinieerd als \verb|\thechoice)| (let op het haakje).

Het commando \verb|\droppoints| is opnieuw gedefinieerd als

\begin{lstlisting}
\def\droppoints{%
  \leavevmode\unskip\nobreak
  {\padded@point@block}%
  %\par
}
\end{lstlisting}

Het commando \verb|\qformat| is opnieuw gedefinieerd als

\begin{lstlisting}
\qformat{\textbf{\tisdexam@opgavespel{} \thequestion{}} (\totalpoints
         \@pointname) \hfill}
\end{lstlisting}


\section{Verwijderde environments}
De \DocClass{exam} class kent de environments \Package{subpart} en
\Package{subsubpart}. Deze zijn verwijderd uit de \DocClass{tisdexam} class.
Ten eerste zijn opgaven met meer dan \'{e}\'{e}n niveau didactisch niet
echt verantwoord en ten tweede is het niet meer mogelijk om het aantal
open vragen en meerkeuzevragen te tellen omdat er mengvormen mogelijk zijn.


\section{Vanilla}
\label{sec:vanilla}
De \DocClass{tisdexam} class dwingt de gebruiker het document op te stellen
in een bepaald format. Niet iedere gebruiker wil dat. De optie
\Option{vanilla} zorgt ervoor dat het document wordt opgemaakt in de vanilla-%
modus. In feite is het nu identiek aan de \DocClass{exam} class met de
mogelijkheid tot het genereren van voorblad, alle andere instellingen blijven
in de originele staat. De gebruiker moet dan zelf alle opmaak instellen
zoals headers, footers, paginanummering en hyperlinks.

De taalopties worden wel herkend en de \Package{babel} package wordt geladen
met de geselecteerde taal.

\section{Gebruikte packages}
In de `non-vanilla mode' worden de volgende packages expliciet geladen:

\texttt{tikz}, \texttt{babel}, \texttt{geometry}

In de `vanilla-mode' worden de volgende packages expliciet geladen:

\texttt{tikz}, \texttt{babel}, \texttt{geometry}

Vanaf versie 3.0 wordt \verb|hyperref| niet meer automatisch geladen.

\section{Meldingen}
Een aantal meldingen kunnen gegenereerd worden, zie hieronder.

\begin{description}\itemsep-3pt
\item[\texttt{Document is typed vanilla}]
  \hfill \\  De optie \verb|vanilla| is geactiveerd.
\item[\texttt{No language specified, set to 'dutch'.}]
   \hfill \\ Er is geen taal geselecteerd, Nederlands is de standaardinstelling.  
\item[\texttt{Subparts not allowed in this class}]
   \hfill \\ Er kunnen geen subparts gebruikt worden in deze class.
\item[\texttt{Subsubparts not allowed in this class}]
  \hfill \\ Er kunnen geen subsubparts gebruikt worden in deze class.
\end{description}



\section{Opgelet!}
De volgende commando's uit de \DocClass{exam} class mogen niet gebruikt
worden: \verb|\checkboxeshook|, \verb|\choiceshook| en \verb|\questionshook|.
Deze worden door de \DocClass{tisdexam} class gebruikt voor interne doeleinden.
Gebruik de nieuwe commando's \verb|\tisdexamcheckboxeshook|, \\
\verb|\tisdexamchoiceshook| en \verb|\tisdexamquestionshook|. Uiteraard geldt
dit niet als de vanilla-modus wordt gebruikt.

De package \verb|geometry| wordt geladen met een standaard instelling. Als
gebruiker kan deze package dus niet geladen worden. Gebruik het commando
\verb|newgeometry| om de pagina-layout in te stellen. Zie ook volgende
hoofdstuk.


\section{Tips}
Het is handig om \verb|hyperref| te laden. De linkkleur is standaard zwart.
Om De kleur in blauw te veranderen, gebruiken we:

\begin{lstlisting}
\begin{document}
\hypersetup{colorlinks,linkcolor=blue}
  ...
\end{lstlisting}

Op een (papieren) toets zijn kleuren misschien niet handig, maar in een
digitale versie met antwoorden weer wel:

\begin{lstlisting}
%% Set link color to black for exam text only.
\ifprintanswers\else\hypersetup{linkcolor=black}\fi
\end{lstlisting}


Standaard is de witruimte tussen het eind van een vraag en het begin van een
nieuwe vraag ingesteld op 0,7 cm. Om deze instelling te wijzigen moet de
lengte \verb|\questionsep| aangepast worden.

\begin{lstlisting}
\setlength{\questionsep}{0.4cm}
\end{lstlisting}


De \verb|geometry| package wordt geladen en de linker en rechter marge worden
ingesteld op 2~cm. De pagina-layout kan opnieuw worden ingesteld
d.m.v.\@ het commando \verb|\newgeometry|.

\begin{lstlisting}
\newgeometry{bindingoffset=0.2in,left=1.0in,right=1.0in,top=0.6in,
             bottom=0.4in,footskip=.2in}
\end{lstlisting}

\section{Voorbeeldvragen}
In dit hoofdstuk wordt een aantal voorbeeldvragen gegeven


\begin{questions}
Voorbeeld van een enkele open vraag:
         
\begin{lstlisting}
\question[10]
Hoeveel is $1+1$?
\end{lstlisting}
\question[10]
Hoeveel is $1+1$?


\bigskip\noindent
Voorbeeld van een open vraag met subvragen:

\begin{lstlisting}
\question
Gegeven de functies: $f(x) = x^3+3x$ en $g(x) = x^2 + 6x$
\begin{parts}
\part[3] \droppoints\ Bepaal de nulpunten van deze functies.
\part[5] \droppoints\ Bepaal de extremen van deze functies.
\part[7] \droppoints\ Bepaal de snijpunten van $f(x)$ met $g(x)$.
\end{parts}
\end{lstlisting}
\question
Gegeven de functies: $f(x) = x^3+3x$ en $g(x) = x^2 +6x$
\begin{parts}
\part[3] \droppoints\ Bepaal de nulpunten van deze functies.
\part[5] \droppoints\ Bepaal de extremen van deze functies.
\part[7] \droppoints\ Bepaal de snijpunten van $f(x)$ met $g(x)$.
\end{parts}

Voorbeeld vraag uit een toets Belastingrecht van het NCOI (examennummer:
96038, dd 23 november 2013) met andere benaming voor
punten.

\begin{lstlisting}
\pointname{ punten}
\marginpointname{ punten}
\question
Ondernemer Hans Bakker (alleenstaand, 30 jaar) heeft al sinds 2003 een
eenmanszaak. Hij verricht bouwkundige inspecties bij zijn klanten. Sinds
2009 heeft hij voor 40 uren per week Klaas Klaver in vaste dienst. Uit de
administratie van Hans over het jaar 2012 blijkt dat hij 1.600 uren heeft
gewerkt. Paula Hartman, de vriendin van Hans, woont 10 kilometer bij hem
vandaan en zij komt Hans sinds 2011 elke zaterdagochtend 3 uren assisteren
met de administratie. Zij ontvangt daarvoor geen enkele vergoeding. Het
betreft een soort van vriendendienst. Behalve dat Hans voor de
inkomstenbelasting belastingplichtig is, moet hij ook voor de omzetbelasting
aangifte doen. Gezien zijn omzet moet hij elk kwartaal zijn btw-aangifte
invullen.

\begin{parts}
\part[8] Welke tabel moet Hans Bakker hanteren in verband met de loonbelasting
van personeelslid Klaas Klaver? Motiveer uw antwoord. {} \droppoints
\part[7] Heeft ondernemer Hans Bakker over 2013 recht op de
zelfstandigenaftrek als zijn situatie gelijk is aan die van 2012? Motiveer uw
antwoord. {} \droppoints
\part[6] Kan ondernemer Hans Bakker over 2013 een bedrag opvoeren in verband
met de meewerkaftrek als zijn situatie gelijk is aan die van 2012? Motiveer
uw antwoord. {} \droppoints
\part[8] Onder welk btw-tarief vallen de werkzaamheden die ondernemer Hans
Bakker verricht? Motiveer uw antwoord. {} \droppoints
\end{parts}
\end{lstlisting}

\pointname{ punten}
\marginpointname{ punten}
\question
Ondernemer Hans Bakker (alleenstaand, 30 jaar) heeft al sinds 2003 een
eenmanszaak. Hij verricht bouwkundige inspecties bij zijn klanten. Sinds
2009 heeft hij voor 40 uren per week Klaas Klaver in vaste dienst. Uit de
administratie van Hans over het jaar 2012 blijkt dat hij 1.600 uren heeft
gewerkt. Paula Hartman, de vriendin van Hans, woont 10 kilometer bij hem
vandaan en zij komt Hans sinds 2011 elke zaterdagochtend 3 uren assisteren
met de administratie. Zij ontvangt daarvoor geen enkele vergoeding. Het
betreft een soort van vriendendienst. Behalve dat Hans voor de
inkomstenbelasting belastingplichtig is, moet hij ook voor de omzetbelasting
aangifte doen. Gezien zijn omzet moet hij elk kwartaal zijn btw-aangifte
invullen.

\begin{parts}
\part[8] Welke tabel moet Hans Bakker hanteren in verband met de loonbelasting
van personeelslid Klaas Klaver? Motiveer uw antwoord. {} \droppoints
\part[7] Heeft ondernemer Hans Bakker over 2013 recht op de
zelfstandigenaftrek als zijn situatie gelijk is aan die van 2012? Motiveer uw
antwoord. {} \droppoints
\part[6] Kan ondernemer Hans Bakker over 2013 een bedrag opvoeren in verband
met de meewerkaftrek als zijn situatie gelijk is aan die van 2012? Motiveer
uw antwoord. {} \droppoints
\part[8] Onder welk btw-tarief vallen de werkzaamheden die ondernemer Hans
Bakker verricht? Motiveer uw antwoord. {} \droppoints
\end{parts}

Voorbeeld van een multiple choice vraag:

\begin{lstlisting}
\question[5]
\label{opg:opg3}
Een gebruiker wil van R25 bit 6 en 1 inverteren en bit 4 en 2 op nul
zetten. Hiervoor zijn een EXOR- en een AND-masker nodig. Welke waardes
zijn correct?
\begin{choices}
	\choice EXOR = 0xBD, AND = 0x14
	\choice EXOR = 0x42, AND = 0x14
	\choice EXOR = 0xBD, AND = 0xEB
	\CorrectChoice \label{ans:opg3} EXOR = 0x42, AND = 0xEB
\end{choices}
\end{lstlisting}

\pointname{ pt}
\marginpointname{ pt}
\question[5]
\label{opg:opg3}
Een gebruiker wil van R25 bit 6 en 1 inverteren en bit 4 en 2 op nul zetten.
Hiervoor zijn een EXOR- en een AND-masker nodig. Welke waardes zijn correct?
\begin{choices}
	\choice EXOR = 0xBD, AND = 0x14
	\choice EXOR = 0x42, AND = 0x14
	\choice EXOR = 0xBD, AND = 0xEB
	\CorrectChoice \label{ans:opg3} EXOR = 0x42, AND = 0xEB
\end{choices}

Bij vraag \ref{opg:opg3} is het label \lstinline|opg:opg3| geplaatst zodat hieraan
gerefereerd kan worden. Bij het goede antwoord is het label \lstinline|ans:opg3|
geplaatst. Hiermee kan het goede antwoord afgedrukt worden.

\begin{lstlisting}
Het goede antwoord op vraag \ref{opg:opg3} is \ref{ans:opg3}.
\end{lstlisting}
Het goede antwoord op vraag \ref{opg:opg3} is \ref{ans:opg3}.

\end{questions}



\end{document}
