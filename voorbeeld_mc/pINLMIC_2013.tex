%
% Test exam document for Course INLMIC (Inleiding Microcontrollers)
% at The Hague University of Applied Sciences, Electrical Engineering.
%
% (c)2013-2017, J. op den Brouw <J.E.J.opdenBrouw@hhs.nl>

%
% This tex document makes use of the exam class.
%


%% 12pt charachters, A4 paper size, one side printing, equation left aligned
\documentclass[a4paper,12pt,fleqn,dutch]{tisdexam}
%\documentclass[a4paper,12pt,fleqn,dutch,answers]{tisdexam}
%                                       ^^^^^^^
%                           remove answers for no answers/solutions


% User adjustable commands
\opleiding{Elektrotechniek}
\opsteller{J.E.J. op den Brouw}
\tweedelezer{J.Z.M. Broeders}
\toetsnaam{INLMIC}
\toetsnaamkort{INLMIC (proeftoets)}
\groep{EQ1}
\toetsdatum{1 januari 1970}
\toetsdatumkort{01-01-1970}
\tijd{0:00 -- 1:30}
\opmerkingen{Beoordeling tentamen: Elk goed antwoord levert 90/\numquestions{} punten op, in totaal zijn 90 punten te behalen. Eindcijfer = 1 + (aantal behaalde punten / 10)}
\module{}
\cesuur{Voldoende bij eindcijfer 5,5 of hoger}

\kladpapiertrue
\antwoordformulierabcdetrue
\grafischerekenmachinetrue
\eigenaantekeningentrue
\boekendictatentrue

%% Use packages...
\usepackage{array}
\setlength{\mathindent}{1em}

%% PDF Version and compression...
\pdfminorversion=5
\pdfobjcompresslevel=2

%% Include graphics files
\usepackage{graphicx}

%% Enumerate items
\usepackage{enumitem}

%% Using floats
\usepackage{float}

%% Used dashed lines in tables
\usepackage{arydshln}

%% Use the AMS Mathematical characters
\usepackage{amsmath}
\usepackage{amsfonts}
\usepackage{amssymb}

%% Set input encoding to ISO-8859-1 (latin1)
\usepackage[latin1]{inputenc}

\ifx \dyslect \undefined
	%% If you want to use the Latin Modern font, uncomment this package
	%% and comment out the Helvetica packages.
	\usepackage{mathptmx}
	\let\mathup\mathrm

\else
	%% Use Helvitica instead of Times
	\usepackage[scaled]{helvet}
	\renewcommand*\familydefault{\sfdefault} %% Only if the base font of the document is to be sans serif
\fi

\usepackage[T1]{fontenc}

%% Making captions nicer...
\usepackage[font=footnotesize,format=plain,labelfont=bf,up,textfont=it,up]{caption}
\captionsetup[table]{justification=raggedright,singlelinecheck=off}

%% Command \overbar zet de bar iets hoger dan \overline
%% Werkt helaas niet genest
\newcommand*{\overbar}[1]{%
  \overline{\mbox{#1}\raisebox{2.54mm}{}}%
}

%% Line over text as in "not" or "inverse"
\newcommand*{\oline}[1]{\overline{#1\mathstrut}}

%% Create a new file (pointer?)
\newwrite\tempfile

%%% No package loading from here

%% Redef \qformat without points after ..
\qformat{\textbf{Opgave \thequestion{}} \hfill}

%% Ahhhh... At last, the beginning of the document...
\begin{document}

%%%%%%%%%%%%%%%%%%%%%%%%%%%%%
%%                         %%
%%   Cover pages of exam   %%
%%                         %%
%%%%%%%%%%%%%%%%%%%%%%%%%%%%%

\makecoverpage

%%%%%%%%%%%%%%%%%%%%%%%%%%%%%
%%                         %%
%%    Instruction Page     %%
%%                         %%
%%%%%%%%%%%%%%%%%%%%%%%%%%%%%
%\raggedright
%\fbox{
	\begin{minipage}[c]{0.979\linewidth}
	{\fontfamily{phv}\fontsize{11pt}{12pt}\selectfont\bfseries
	{\addtolength{\leftskip}{6mm} SCHRAPINSTRUCTIE TOETSKAART TYPE ABCDE: \par}
	\par}%

	{\fontfamily{phv}\fontsize{11pt}{12pt}\selectfont
		\begin{itemize}\itemsep-4pt
			\item Vul bij StudentNummer je studentnummer in met met zwarte of blauwe pen.
			\item Vul ook in de blokjes eronder je StudentNummer in.
			\item Vul bij de vragen het antwoord in met zwarte of blauwe pen.
			\item Vul bij Tentamen de naam van het vak waarover de toets gehouden wordt.
			\item Vul bij Naam je naam en voorletters in.
			\item Bij dag / maand / jaar de datum van de toets.
			\item Vul het versienummer in (indien van toepassing).
			\item Eventuele opmerkingen kun je onderaan de pagina vermelden.
		\end{itemize}
		\vskip 1.0cm
	\par}%
	\end{minipage}\hfill
%}

\begin{figure}[h!]
  \centering
  \includegraphics*[viewport=0 500 600 850,scale=0.80]{vijfkeuze_nl.pdf}
\end{figure}
\centering{***********************************************************}
\begin{figure}[h!]
  \centering
  \includegraphics*[viewport=0 0 600 200,scale=0.80]{vijfkeuze_nl.pdf}
\end{figure}

{\fontfamily{phv}\fontsize{11pt}{12pt}\selectfont\bfseries
\centering{\large{\textbf{Vraag een exemplaar aan de surveillant}}}
\par}

\vspace*{-10cm}
{\fontfamily{phv}\fontsize{11pt}{12pt}\selectfont\bfseries
\Huge{\textbf{Voorbeeld}}
\Huge{\textbf{Voorbeeld}}

\Huge{\textbf{Voorbeeld}}
\Huge{\textbf{Voorbeeld}}

\Huge{\textbf{Voorbeeld}}
\Huge{\textbf{Voorbeeld}}

\Huge{\textbf{Voorbeeld}}
\Huge{\textbf{Voorbeeld}}
\par}
\newpage

\renewcommand{\arraystretch}{1.1}

%%%%%%%%%%%%%%%%%%%%%%%%%%%%%
%%                         %%
%% First real page of exam %%
%%                         %%
%%%%%%%%%%%%%%%%%%%%%%%%%%%%%

\begin{questions}


\question
\label{opg:opg1}
R18 heeft voor het uitvoeren van onderstaande instructie de binaire waarde 0011 0110.
Wat is het resultaat na het uitvoeren van de volgende instructie?
\begin{verbatim}
    ori r18, 0xA4
\end{verbatim}
\begin{choices}
	\choice R18 bevat het decimale getal 64.
	\choice R18 bevat het decimale getal 22.
	\CorrectChoice \label{ans:opg1} R18 bevat het hexadecimale getal B6.
	\choice R18 bevat het hexadecimale getal FC.
\end{choices}


\question
\label{opg:opg2}
Een gebruiker wil van R25 bit 7 en 6 op \'{e}\'{e}n zetten en bit 3 en 2 op nul. Hiervoor zijn een
AND- en een OR-masker nodig. Welke waardes zijn correct?
\begin{choices}
	\choice AND = 0x3F, OR = 0xC0
	\CorrectChoice \label{ans:opg2} AND = 0xF3, OR = 0xC0
	\choice AND = 0xC0, OR = 0xF3
	\choice AND = 0xF3, OR = 0x0C
\end{choices}


\question
\label{opg:opg3}
De pinnen van Port D zijn geconfigureerd als inputs, de pinnen van Port B zijn geconfigureerd als
outputs. Op de pinnen van Port D wordt de waarde 0x34 gezet.
Wat is de waarde die op de pinnen van Port B staat na uitvoer van de volgende instructies:
\begin{verbatim}
    in  r20,PIND
    lsl r20
    lsl r20
    out PORTB,r20
\end{verbatim}
\begin{choices}
	\choice 0xF4
	\choice 0xE4
	\choice 0xE8
	\CorrectChoice \label{ans:opg3} 0xD0
\end{choices}


\question
\label{opg:opg4}
De inhoud van R1 is na het uitvoeren van de instructie:
\begin{verbatim}
    eor r1,r1
\end{verbatim}
\begin{choices}
	\choice Verdubbeld.
	\choice Ge\"{i}nverteerd.
	\CorrectChoice \label{ans:opg4} Nul.
	\choice Gehalveerd.
\end{choices}


\question
\label{opg:opg5}
De individuele pinnen van Port D kunnen als ingang of uitgang geconfigureerd worden. De
enkelvoudige bits van Port D worden PD7, PD6, PD5, PD4, PD2, PD2, PD1, PD0 genoemd.
Welk van de volgende alternatieven configureert PD0, PD1, PD6 en PD7 als uitgang en de
overige pinnen als ingang?
\begin{choices}
	\choice \begin{verbatim}out DDRD,0x3c\end{verbatim}
	\CorrectChoice \label{ans:opg5} \begin{verbatim}ldi r20,0xc3
out DDRD,r20\end{verbatim}
	\choice \begin{verbatim}out PORTD,0x3c\end{verbatim}
	\choice \begin{verbatim}ldi r20,195
out PORTD,r20\end{verbatim}
\end{choices}


\question
\label{opg:opg6}
Welke bit(s) in het statusregister SREG zijn gezet na uitvoeren van de volgende
instructies, als vooraf alle bits in het SREG op '0' ge\"{i}nitialiseerd zijn?
\begin{verbatim}
    ldi r16,0x4f
    ldi r17,0x3f
    add r16,r17
\end{verbatim}
\begin{choices}
	\choice geen van de bits
	\choice het Z-bit
	\choice het N-bit
	\CorrectChoice \label{ans:opg6} het N-bit en V-bit
\end{choices}


\question
\label{opg:opg7}
Bij een nieuwe microcontroller zijn de ontwerpers vergeten een ADC-instructie te maken. Er
bestaat alleen een ADD-instructie die de carry niet meeneemt in de optelling maar wel wordt de
carrybit gezet in het statusregister SREG gezet indien er een overloop is. Hieronder volgen twee
stukjes code met 16-bit optellingen waarbij de carry toch wordt opgeteld. Ga ervan uit dat de 16-
bit optelling als geheel geen overloop geeft en ga uit van unsigned getallen. Welke stukje(s)
is/zijn correct?
\begin{verbatim}
               ; code I                ; code II
               add  r20,r24            add  r20,r24
               brcc over1              add  r21,r25
               subi r21,-1             brcc over2
        over1: add  r21,r25            subi r21,-1
               ...              over2: ...
\end{verbatim}
\begin{choices}
	\choice Zowel code I als code II zijn onjuist
	\choice Code I is onjuist en code II is juist.
	\CorrectChoice \label{ans:opg7} Code I is juist en code II is onjuist
	\choice Zowel code I als code II zijn juist.
\end{choices}


\question
\label{opg:opg8}
Pin 1 van port C is momenteel ingesteld als uitgang met als waarde 0, deze pin ligt dus hard
aan de 0 volt. Met welke instructie-reeks kan je hier veilig een Pullup van maken?
\begin{choices}
	\CorrectChoice \label{ans:opg8} \begin{verbatim}cbi DDRC,1    sbi PORTC,1\end{verbatim}
	\choice \begin{verbatim}cbi DDRC,1    sbi PINC,1\end{verbatim}
	\choice \begin{verbatim}sbi PORTC,1   cbi DDRC,1\end{verbatim}
	\choice \begin{verbatim}sbi PINC,1    cbi DDRC,1\end{verbatim}
\end{choices}


\question
\label{opg:opg9}
Een ontwerper wil graag afwisselend een rode en groene led laten branden. Hiervoor heeft zij
onderstaande oplossing bedacht. Pin 3 van Port B wordt hiervoor gebruikt. Welke van de
onderstaande mogelijkheden zorgt er voor dat de groene led gaat branden?
\begin{figure}[H]
  \centering
  	%\fbox{
    \includegraphics*[viewport=270 180 500 490,scale=0.55]{pINLMIC_led_schak.pdf}
   	%}
  %\caption{Blokschema machine.}
  %\label{fig:opgave2}
\end{figure}
\begin{choices}
	\CorrectChoice \label{ans:opg9} \begin{verbatim}sbi DDRB,3 cbi PORTB,3\end{verbatim}
	\choice \begin{verbatim}sbi DDRB,3 sbi PORTB,3\end{verbatim}
	\choice \begin{verbatim}cbi DDRB,3 cbi PORTB,3\end{verbatim}
	\choice \begin{verbatim}cbi DDRB,3 sbi PORTB,3\end{verbatim}
\end{choices}


\question
\label{opg:opg10}
Een ATmega32 microcontroller is aangesloten op een 8 MHz kristal.
Als we met onderstaand programma een wachttijd van 100 milliseconde
willen realiseren op welke waarden moeten de registers delay1, delay2 en
delay3 dan worden ge\"{i}nitialiseerd?
\begin{verbatim}
    loop: subi delay1,1
          sbci delay2,0
          sbci delay3,0
          brcc loop
\end{verbatim}
\begin{choices}
	\choice Delay1 = 0x30, Delay2 = 4B, Delay3 = 04
	\CorrectChoice \label{ans:opg10} Delay1 = 0x{f}{f}, Delay2 = 70, Delay3 = 02
	\choice Delay1 = 0x{f}{f}, Delay2 = 34, Delay3 = 0B
	\choice Anders dan opgegeven in a, b of c.
\end{choices}


\question
\label{opg:opg11}
Wat is de maximale wachttijd die met het programma van de vorige opgave kan worden
gerealiseerd?
\begin{choices}
	\choice 4,1 seconden
	\choice 21,0 seconden
	\choice 6,7 seconden
	\CorrectChoice \label{ans:opg11} 10,5 seconden
\end{choices}


\question
\label{opg:opg12}
Bekijk de volgende code. Welke uitspraak is correct?
\begin{verbatim}
    ldi r16,0b01100000
    out GICR,r16
    ldi r16,0b00001100
    out MCUCR,r16
\end{verbatim}
\begin{choices}
	\CorrectChoice \label{ans:opg12} INT0 is geactiveerd.
	\choice INT1 reageert op beide flanken
	\choice INT0 wordt geactiveerd als flankgevoelige interrupt.
	\choice INT1 is geactiveerd en reageert op een opgaande flank
\end{choices}


\question
\label{opg:opg13}
De uitkomst van een rekenkundige bewerking kan nul opleveren, dus de uitgangen van de
ALU zijn alle 0. De Z-flag van het Status Register wordt dan 1. Dit kan geregeld worden met een
logische functie. Zie onderstaande figuur. Welke logische functie is dit (er zijn geen
geinverteerde signalen beschikbaar)?
\begin{figure}[H]
  \centering
  	%\fbox{
    \includegraphics*[viewport=130 170 540 410,scale=0.55]{pINLMIC_zero_flag.pdf}
   	%}
  %\caption{Blokschema machine.}
  %\label{fig:opgave2}
\end{figure}
\begin{choices}
	\choice AND
	\choice EXOR
	\CorrectChoice \label{ans:opg13} NOR
	\choice NAND
\end{choices}


\newpage
\question
\label{opg:opg14}
Bekijk de volgende code
\begin{verbatim}
    ldi r16,0b00011011
    out TCCR0,r16
\end{verbatim}
\begin{choices}
	\choice T/C0 staat in de normal mode met prescaler op 1024
	\CorrectChoice \label{ans:opg14} T/C0 staat in de CTC-mode met prescaler 64 en OC0 op Toggle On Compare Match
	\choice T/C0 staat in de normal mode met prescaler 256 en OC0 op Set On Compare Match
	\choice T/C0 staat anders ingesteld dan aangegeven in a, b of c.
\end{choices}


\question
\label{opg:opg15}
Een ATmega32 wordt geklokt op 7,3728 MHz. T/C0 moet elke 16,67 ms een OCM-interrupt
genereren. De waarde voor de prescaler en OCR0 zijn dan:
\begin{choices}
	\choice prescaler = 256, OCR0 = 480
	\CorrectChoice \label{ans:opg15} prescaler = 1024, OCR0 = 119
	\choice prescaler = 64, OCR0 = 100
	\choice prescaler = 1024, OCR0 = 255
\end{choices}


\question
\label{opg:opg16}
Een ATmega32 loopt op 3,579545 MHz. Wat bedraagt de tijd tussen twee overflowinterrupts
van T/C0 als prescalerwaarde 256 wordt gebruikt?
\begin{choices}
	\CorrectChoice \label{ans:opg16} 18,3 ms
	\choice 10,1 ms
	\choice 54,6 ms
	\choice 23,5 ms
\end{choices}


\question
\label{opg:opg17}
Hieronder is het blokschema van een eenvoudige processor te vinden. De registers werken op
hetzelfde kloksignaal. De inhouden van A en B zijn onbekend. Op de ingang 'data' wordt
continu het getal 5 aangeboden. De gebruiker stuurt zelf de besturingssignalen aan (muxa...).
\begin{figure}[H]
  \centering
  	%\fbox{
    \includegraphics*[viewport=70 110 500 460,scale=0.55]{pINLMIC_eenvoudige_rekenaar.pdf}
   	%}
  %\caption{Blokschema machine.}
  %\label{fig:opgave2}
\end{figure}
Ontwerp nu een 'programma' dat met een minimum aan 'instructies' (aansturen van de
besturingssignalen) de waarde 35 in register B plaatst. Elke instructie kost \'{e}\'{e}n
klokpuls. Dit kost dan minimaal:
\begin{choices}
	\choice 4 klokpulsen
	\choice 5 klokpulsen
	\CorrectChoice \label{ans:opg17} 6 klokpulsen
	\choice 7 klokpulsen
\end{choices}


\question
\label{opg:opg18}
De ATmega32 microcontroller heeft 21 mogelijke interrupts. Als twee interrupts gelijktijdig
optreden, hoe verloopt dan de afhandeling van de interrupts?
\begin{choices}
	\choice Een interne interrupt wordt als eerste afgehandeld.
	\choice Een externe interrupt wordt als eerste afgehandeld.
	\choice De interrupt met het hoogste adres in de vectortabel wordt als eerste afgehandeld.
	\CorrectChoice \label{ans:opg18} De interrupt met het laagste adres in de vectortabel wordt als eerste afgehandeld.
\end{choices}


\question
\label{opg:opg19}
Gegeven de volgende code:
\begin{verbatim}
    .nolist
    .include "m32def.inc"
    .list
    .org 0x000
            ldi  r16,low(RAMEND)
            out  SPL,r16
            ldi  r16,high(RAMEND)
            out  SPH,r16
            ldi  r19,0xff
            ldi  r18,13
            call multi
    halt:   rjmp halt

    multi:  push r19
            push r18
            lsl  r18
            lsl  r18
            pop  r19
            add  r18,r19
            lsl  r18
            pop  r19
            ret
\end{verbatim}
\newpage
Er worden drie uitspraken gedaan:
\begin{enumerate}[itemsep=-1pt,leftmargin=23pt,label=\Roman*.]
\item De instructie \texttt{lsl} schuift de inhoud van een register naar rechts.
\item Na het uitvoeren van \texttt{multi} is R18 vermenigvuldigd met 10.
\item R19 heeft na het uitvoeren van de eerste \texttt{pop}-instructie de waarde 0x{f}{f}.
\end{enumerate}
\begin{choices}
	\CorrectChoice \label{ans:opg19} I is niet waar en II is waar
	\choice II is waar en III is waar
	\choice I is waar en II en III zijn niet waar
	\choice I is niet waar en III is waar
\end{choices}


\question
\label{opg:opg20}
Welke van de volgende uitspraken is juist/onjuist?
\begin{enumerate}[itemsep=-1pt,leftmargin=23pt,label=\Roman*.]
	\item De vlaggen kunnen geladen worden in R16.
	\item DDRC kan in R16 worden geladen d.m.v. een \texttt{lds}-instructie.
\end{enumerate}
\begin{choices}
	\choice Zowel I als II zijn onjuist
	\choice I is onjuist en II is juist
	\choice I is juist en II is onjuist
	\CorrectChoice \label{ans:opg20} Zowel I als II zijn juist.
\end{choices}


\question
\label{opg:opg21}
Hieronder is het schema van een eenvoudige microcontroller zoals in de les behandeld te
zien. Alle geklokte onderdelen krijgen hetzelfde kloksignaal aangeboden.
\begin{figure}[H]
  \centering
  	%\fbox{
    \includegraphics*[viewport=100 50 540 550,scale=0.55]{pINLMIC_avr_thr_v4.pdf}
   	%}
  %\caption{Blokschema machine.}
  %\label{fig:opgave2}
\end{figure}


Men wil register 13 laden met de AND van register 3 en register 15 en verder geen andere
door de gebruiker geprogrammeerde acties. Welke van de onderstaande
instellingen zorgt daarvoor? Hint: in de sheets staan diverse tabellen.\\
\begin{verbatim}    acode muxrsr muxrsl muxrd enrd bcode paddr\end{verbatim}
\begin{choices}
	\choice \begin{verbatim}0100  00011  01111  01101 1    111   0000000000\end{verbatim}
	\choice \begin{verbatim}0010  10010  10110  01100 0    011   0101010101\end{verbatim}
	\CorrectChoice \label{ans:opg21} \begin{verbatim}0100  00011  01111  01101 1    000   1010101010\end{verbatim}
	\choice \begin{verbatim}0001  10010  00111  10010 1    001   1111111111\end{verbatim}
\end{choices}


\question
\label{opg:opg22}
\label{opg:rettrick}
Bij het aanroepen van een subroutine wordt het terugkeeradres op de stack opgeslagen. Dit
terugkeeradres bestaat bij de ATmega32 uit twee bytes. Het lage byte van het terugkeeradres
wordt eerst op de stack gezet, het hoge byte als tweede. Een gebruiker heeft onderstaand
programma geschreven.
\begin{verbatim}
    .nolist
    .include "m32def.inc"
    .list
    .org 0x000
              ldi r16,low(RAMEND)
              out SPL,r16
              ldi r16,high(RAMEND)
              out SPH,r16
              call rettrick
    loop:     rjmp loop
    
    rettrick: pop r16
              pop r16
              ldi r16,0x01
              push r16
              ldi r16,0x05
              push r16
              ret

    .org 0x0105
    halt1: rjmp halt1
    
    .org 0x0501
    halt2: rjmp halt2
\end{verbatim}
Na uitvoeren van de \texttt{ret}-instructie wordt gesprongen naar de instructie direct achter het label
\begin{choices}
	\choice loop
	\choice rettrick
	\choice halt1
	\CorrectChoice \label{ans:opg22} halt2
\end{choices}
Hint: teken de stack tijdens uitvoering van de \texttt{CALL}-instructie en de routine \texttt{rettrick}.


\question
\label{opg:opg23}
Bestudeer het programma uit vraag \ref{opg:rettrick} aandachtig. Het Flash-ROM adres waar routine
rettrick begint is (in words). Hint: slides geven aanwijzingen over de grootte van instructies.
\begin{choices}
	\choice 0x006
	\CorrectChoice \label{ans:opg23} 0x007
	\choice 0x008
	\choice 0x009
\end{choices}


\question
\label{opg:opg24}
Bestudeer het programma uit vraag \ref{opg:rettrick} aandachtig. De
stackpointer wordt ge\"{i}nitialiseerd 0x085f. Wat is het laagste adres dat de
stackpointer aanwijst tijdens het draaien van deze code?
\begin{choices}
	\CorrectChoice \label{ans:opg24} 0x085d
	\choice 0x085c
	\choice 0x0860
	\choice 0x085f
\end{choices}






%\vskip5cm
%
%\begin{center}
%*-*-*-*-*-*-*     einde toets     *-*-*-*-*-*-*     
%\end{center}

%%%%%%%%%%%%%%%%%%%%%%%%%%%%%%%%%%%%%%%%%%
%%% Answers to the questions           %%%
%%%%%%%%%%%%%%%%%%%%%%%%%%%%%%%%%%%%%%%%%%

\ifprintanswers
\newpage
\textbf{Uitwerking opgaven}
\vspace{1cm}

Opgave \ref{opg:opg1}. R18 heeft de binaire waarde 0011 0110. Dit moet ge-OR-d worden
met 0xA4, binair is dat 1010 0100.

\begin{table}[h!]
	\begin{tabular}{l l l l l}
		 & \texttt{0011 0110}      &  &  0x36  &  54   \\ 
	 	 & \texttt{1010 0100 or}   &  &  0xA4  &  164  \\ 
		 & \texttt{{-}{-}{-}{-}{-}{-}{-}{-}{-}{-}{-}{-}}  &  &        &       \\ 
		 & \texttt{1011 0110}      &  &  0xB6  &  182  \\
	\end{tabular} 
\end{table}

Het resultaat is 182 (decimaal) of 0xB6 (hexadecimaal). Zie de sheets week 4,
pagina 12 - 15. Het antwoord is~\ref{ans:opg1}.

\vspace{1em}
Opgave \ref{opg:opg2}. Hier wordt een combinatie van zetten en wissen van bits
gevraagd. We noteren eerst wat er moet gebeuren: een 1 is zetten, een 0 is
wissen en een - betekent ongewijzigd laten. Een OR levert 1-en een AND levert
0-en. Zie de sheets week 3, pagina 40 - 43.

\begin{table}[h!]
	\begin{tabular}{l l l l}
		 & \texttt{7654 3210}  &  &  bitnummers   \\ 
	 	 & \texttt{11-- 00--}  &  &  wat moet er gebeuren \\ 
		 & \texttt{1100 0000}  &  &  OR-masker (1 is zetten, 0 is ongemoeid laten)    \\ 
		 & \texttt{1111 0011}  &  &  AND-masker  (0 is wissen, 1 is ongemoeid laten)  \\
	\end{tabular} 
\end{table}

Het resultaat is dus AND = 0xF3 en OR = 0xC0. Het antwoord is~\ref{ans:opg2}.

\vspace{1em}
Opgave \ref{opg:opg3}. Op de pinnen van PORTD wordt de waarde 0x34 gezet. Dit is
binair 0011 0100. In onderstaande code wordt dit patroon eerst ingelezen in
R20. Daarna wordt er twee keer linksom geschoven. Zie boek blz 51. Zie
onderstaande code en commentaar.

\begin{table}[h!]
	\begin{tabular}{l l l l}
		 & \texttt{in  r20,PIND}   &  &  R20 = 0011 0100  \\ 
	 	 & \texttt{lsl r20}        &  &  R20 = 1111 0100  \\ 
		 & \texttt{lsl r20}        &  &  R20 = 1110 1000  \\ 
		 & \texttt{out PORTB,r20}  &  &   \\
	\end{tabular} 
\end{table}

Het resultaat is 0xD0. Het antwoord is dan~\ref{ans:opg3}.

\vspace{1em}
Opgave \ref{opg:opg4}. Bij  de instructie \texttt{eor ra,rb} worden de bits van
register a en register b bitsgewijs ge-exor-d. Hieronder is de bekende tabel
gegeven (n staat voor bitnummer 0 t/m 7).

\begin{table}[h!]
	\begin{tabular}{c c | c}
		\hline
		 $ra_{n}$ & $rb_{n}$ &  $F_{n}$  \\ \hline
		    0     &    0     &    0      \\ 
		    0     &    1     &    1      \\
		    1     &    0     &    1      \\
		    1     &    1     &    0      \\ \hline
	\end{tabular} 
\end{table}
			
Nu wordt gevraagd wat de het resultaat is van de instructie \texttt{eor r1,r1}.
Hierin wordt R1 ge-exor-d met zichzelf! Dus als bit 0 van R1 een 0 is wordt dit
bit ge-exor-d met 0. En dat levert als resultaat een 0. Precies hetzelfde
resultaat wordt verkregen als het bit een 1 is. Met andere woorden: alle bits
worden 0! Alleen de eerste en laatste regel van de tabel zijn van toepassing.
Dit is een bekende truc om een register op 0 te krijgen als een processor geen
\texttt{clr}-instructie (clear) heeft. Het antwoord is~\ref{ans:opg4}.

\vspace{1em}
Opgave \ref{opg:opg5}. Willen we poortpinnen als uitgang gebruiken dan moet het DDR-register worden
geprogrammeerd met enen op de bitposities van de pinnen die als uitgang moeten
dienen. Aangezien de pinnen 0, 1, 6 en 7 als uitgang moeten worden
geprogrammeerd wordt het bitpatroon 1100.0011 en dit is gelijk aan 0xc3.
I/O-registers programmeer je met een \texttt{out}-instructie, maar dat kan alleen
vanuit een general purpose register, niet direct met een constante, dus je hebt
(bijvoorbeeld) een \texttt{ldi}-instructie nodig om een register met een
constante te laden. Zie sheets week 3, pagina 28 - 29.

Het anwoord is~\ref{ans:opg5}.

\vspace{1em}
Opgave \ref{opg:opg6}. De \texttt{add}-instructie telt twee registers bij elkaar op waarbij de
vlaggen worden aangepast en het resultaat wordt opgeslagen. Zie sheets week 3,
pagina 46.

\begin{table}[h!]
	\begin{tabular}{l l l l}
		 & \texttt{ldi r16,0x4f}   &  &  R16 = 0x4f  \\ 
	 	 & \texttt{ldi r17,0x3f}   &  &  R17 = 0x3f  \\ 
		 & \texttt{add r16,r17}    &  &  R16 = 0x4f + 0x3f = 0x8e ($>$ 0x7f dus overflow!) \\ 
	\end{tabular} 
\end{table}

Nu is 0x4f + 0x3d = 0x8e $>$ 0x7f, dus past niet in een 2's complement 8-bit
register, dus de V-flag wordt op 1 gezet. Het msb is 1 dus de N-flag wordt ook
op 1 gezet. Beschouw je de inhouden van de registers als unsigned, dat past het
resultaat wel, dus de C-flag is 0. De Z-flag is natuurlijk 0.

Het antwoord is~\ref{ans:opg6}.

\vspace{1em}
Opgave \ref{opg:opg7}. De microcontroller heeft geen \texttt{adc}-instructie maar die
kunnen we wel nabootsen (bij multibyte optellingen $>$ 2 bytes wordt het
lastig!). De truc is om eerst de minst significante bytes op te tellen,
daaruit volgt een carry (die kan 0 of 1 zijn). Indien de carry 1 is, moet er
bij de meest significante bytes \'{e}\'{e}n worden opgeteld (dat is immers de
werking van de carry). Dus als de carry 1 is verhogen we de meest significante
bytes met \'{e}\'{e}n en anders niet. Dit is een werkje voor de
\texttt{brcc}-instructie. Als laatste moeten de twee meest significante bytes
worden opgeteld. De eventuele carry die daaruit volgt, verwaarlozen we. Als je
de code bekijkt, zie je dat er twee register-"paren" zijn: R21 met R20 en R25
met R24.

Code I voldoet. Als eerste wordt een \texttt{add}-instructie uitgevoerd die R20
en R24 bij elkaar optelt. Indien de carry 0 is, wordt over de
\texttt{subi}-instructie gesprongen. Is de carry 1, dan "springt" de
\texttt{brcc}-instructie niet en wordt de \texttt{subi}-instructie uitgevoerd,
die 1 bij R21 optelt. Daarna wordt in beide gevallen R25 bij R21 opgeteld.

Code II werkt niet correct. Immers de tweede \texttt{add}-instructie verknoeit
de waarden van de vlaggen die uit de eerste \texttt{add}-instructie zijn
voortgekomen, dus ook de carry-flag.

Het antwoord is~\ref{ans:opg7}.

\vspace{1em}
Opgave \ref{opg:opg8}. Zie sheets week 3, pagina 36, derde item. DDR eerst naar 0, dan
PORT naar 1. De \texttt{cbi}- en \texttt{sbi}-instructies worden uitgelegd op
pagina 39.

Het antwoord is~\ref{ans:opg8}.

\vspace{1em}
Opgave \ref{opg:opg9}. Dit is een grappige schakeling waarbij je twee leds kan
aansturen als ze niet tegelijkertijd moeten branden. Aangezien de ATmega32
zowel 20 mA kan leveren als opnemen, geeft dit geen problemen. Om de groene
led te laten branden moet PB3 natuurlijk als een harde 0 geprogrammeerd zijn,
dus DDRB3 = 1 en PORTB3 = 0. Zie sheets week 3, pagina 29.

Nog veel leuker is dat PB3 ook gebruikt kan worden in de Output Compare Match
modus, die je kan laten toggelen. Daarmee kan je de beide leds (ogenschijnlijk)
op halve intensiteit laten branden. In de PWM-modi kan je nog veel leukere
dingen doen. Zie sheets week 5, pagina~30~--~34.

Het antwoord is~\ref{ans:opg9}.

\vspace{1em}
Opgave \ref{opg:opg10}. Hiervoor moet je de formule gebruiken voor het bepalen van het
aantal keer dat de lus doorlopen moet worden, zie sheets week 4, pagina 9.
\begin{equation}
\nonumber aantal\_lussen = \dfrac{f_{cpu} \cdot wachttijd}{klolpulsen/lus}=
                           \dfrac{8000000 \cdot 0,1}{5} = 160000
\end{equation} 

Daar moet je 1 vanaf trekken, want door de \texttt{brcc}-instructie wordt er
altijd \'{e}\'{e}n keer meer gelust! Dus het getal dat gebruik moet worden is
159999 en dat is 0x0270{f}{f}. Dan wordt delay1 = 0x{f}{f}, delay2 = 0x70 en 
delay3 = 0x02.

Het antwoord is~\ref{ans:opg10}.

\vspace{1em}
Opgave \ref{opg:opg11}. Om wachtlus maximaal tijd te laten verstoken, moet je het
lus-aantal maximaal zien te krijgen. Daarvoor moeten delay1, delay2 en delay3
geladen worden met 0x{f}{f}, hoger kan niet. Het aantal keer dat de lus dan
doorlopen wordt is 0x{f}{f}{f}{f}{f}{f} + 1 (let op de +1). Dit is 16777216 keer. Nu
verbouwen we de de formule uit de slides van week 4.
\begin{equation}
\nonumber aantal\_lussen = \dfrac{f_{cpu} \cdot wachttijd}{klolpulsen/lus} 
                           \Longleftrightarrow wachttijd = 
                           \dfrac{aantal\_lussen \cdot klolpulsen/lus}{f_{cpu}}
\end{equation} 
\begin{equation}
\nonumber wachttijd = \dfrac{16777216 \cdot 5}{8000000} = 10,485... 
                      \approx 10,5 \: \mathup{s}
\end{equation} 

Het antwoord is~\ref{ans:opg11}.

\vspace{1em}
Opgave \ref{opg:opg12}. Zoek eerst de registers en de betekenissen op in de sheets
week 5, pagina 16 - 21. Hieronder de code met commentaar.

\begin{table}[h!]
	\begin{tabular}{l l l l}
		 & \texttt{ldi r16,0b01100000}   &  &  INT0 en INT2 actief  \\ 
	 	 & \texttt{out GICR,r16}         &  &    \\ 
		 & \texttt{ldi r16,0b00001100}   &  &  INT1 op flank, INT0 op level \\ 
	 	 & \texttt{out MCUCR,r16}        &  &    \\ 
	\end{tabular} 
\end{table}

A is correct want INT0 wordt geactiveerd. B is niet correct want INT1 is
ingesteld op opgaande flank (en reageert helemaal niet, wat het staat uit).
C is niet correct want INT0 is geactiveerd als niveau-interrupt. D is niet
correct want INT1 is niet geactiveerd. Het antwoord is~\ref{ans:opg12}.

\vspace{1em}
Opgave \ref{opg:opg13}. De Z-flag kan eenvoudig worden bepaald door alle uitkomstbits
in \'{e}\'{e}n NOR-poort te stoppen. De formule is dus:
\begin{equation}
\nonumber Z = \oline{R7+R6+R5+R4+R3+R2+R1+R0}
\end{equation} 

Je kan het ook anders defini\"{e}ren: Z wordt 1 als alle uitkomstbits 0 zijn.
Dat is in formulevorm:
\begin{equation}
\nonumber Z = \oline{R7}\cdot\oline{R6}\cdot\oline{R5}\cdot\oline{R4}\cdot
                        \oline{R3}\cdot\oline{R2}\cdot\oline{R1}\cdot\oline{R0}
\end{equation}

Maar hiervoor heb je de ge\"{i}nverteerde uitgangen nodig. Met behulp van De
Morgan kan je hieruit naar de bovenste formule omwerken. Zie sheets week 3,
pagina 18.

Het antwoord is~\ref{ans:opg13}.

\vspace{1em}
Opgave \ref{opg:opg14}. Bekijk de volgende code

\begin{table}[h!]
	\begin{tabular}{l l l l}
		 & \texttt{ldi r16,0b00011011}   &  &  T/C0 in CTC mode, prescaler op 64  \\ 
	 	 & \texttt{out TCCR0,r16}        &  &  maar ook OCM-mode op Toggle on Compare Match  \\ 
	\end{tabular} 
\end{table}

Zie ook sheets week 5, pagina 31 - 37. Het antwoord is~\ref{ans:opg14}.

\vspace{1em}
Opgave \ref{opg:opg15}. Deze vraag kan worden opgelost met de formule uit de sheets
week 5, pagina 40. Hiervoor moet de frequentie van het OCM-interrupt signaal
worden uitgerekend, want het staat als tijd genoteerd. Een periodetijd van
16,67 ms is gelijk aan een frequentie van 60 Hz (frequentie van het lichtnet
in Amerika). De prescaler moet op 1024, de andere mogelijkheden leveren geen
goede waarden voor N en OCR0.
\begin{equation}
\nonumber prescaler \cdot N = \dfrac{f_{clk}}{f_{ocm}} 
                    \longrightarrow N = \dfrac{f_{clk}}{f_{ocm} \cdot prescaler}
			= \dfrac{7372800}{60 \cdot 1024} = 120
\end{equation} 

OCR0 is dan 119. Het antwoord is~\ref{ans:opg15}.

\vspace{1em}
Opgave \ref{opg:opg16}. Ook hier wordt een tijd gevraagd, terwijl de formule van
frequentie uitgaat. De frequentie van het kristal is erg nauwkeurig gegeven
en je kan deze ook echt kopen! Zie sheets week 5, pagina 38.
\begin{equation}
\nonumber f_{o} = \dfrac{\dfrac{f_{clk}}{prescaler}}{256} = 
                  \dfrac{f_{clk}}{prescaler \cdot 256} = 54,6
\end{equation}

De frequentie is 54,6 Hz, dus de periodetijd is 18,3 ms.
Het antwoord is~\ref{ans:opg16}.

\vspace{1em}
Opgave \ref{opg:opg17}. Hieronder is wederom het blokschema van een eenvoudige
processor te vinden. De registers werken op hetzelfde kloksignaal. De inhouden
van A en B zijn onbekend. Op de ingang 'data' wordt continue het getal 5
aangeboden. Er zijn twee klokpulsen ("slagen") nodig om beide registers met 5
geladen te hebben, immers alleen A kan data van buitenaf ontvangen. Daarna is
het een kwestie van smaak. Hieronder zie je een mogelijkheid, er zijn ook
andere. Je moet goed opletten dat jouw mogelijk ook echt kan. Het is
bijvoorbeeld niet mogelijk om de inhoud van register B in register A te laden,
omdat daar geen hardwarevoorziening voor is. Na de $6^{e}$ klokpuls heeft
register B de waarde 35. Zie sheets week 1, pagina 37 en 38.

\begin{figure}[H]
  \begin{minipage}[!t]{0.60\linewidth}
    \centering
    %\fbox{
	\includegraphics*[viewport=70 110 500 460,scale=0.55]{pINLMIC_eenvoudige_rekenaar.pdf}
	%}
  \end{minipage}\hfill
  \begin{minipage}[!t]{.350\linewidth}
  	% Tell Latex it's a table
    \begin{tabular}{ c | c c }
      \hline
                &       &               \\ [-2.9ex]
       Klokpuls &  A    &  B     \\ \hline
           0    &  ?    &  ?     \\
           1    &  5    &  ?     \\
           2    &  5    &  5     \\
           3    &  5    &  10    \\
           4    &  15*  &  15*   \\
           5    &  5**  &  30    \\
           6    &  5    &  35    \\ \hline
    \end{tabular}
    
    * A en B tegelijk laden.
    
    ** A wordt weer geladen met 5.
    
    Er zijn andere mogelijkheden.
    \vskip10pt
  \end{minipage}\hfill
\end{figure}

Het antwoord is~\ref{ans:opg17}.

\vspace{1em}
Opgave \ref{opg:opg18}. Als tegelijk twee of meerdere interrupts worden aangevraagd,
wordt degene met het laagste adres in de vector tabel als eerste uitgevoerd.
Zie sheets week 5, pagina 13. 

Het antwoord is~\ref{ans:opg18}.

\vspace{1em}
Opgave \ref{opg:opg19}. We bekijken alleen subroutine \texttt{multi}, de rest in niet
van belang. Hieronder nog de code. Achter de code ook gelijk commentaar.
\begin{verbatim}
1   multi:  push r19        ; R19 later nodig
2           push r18        ; R18 even saven, later weer nodig
3           lsl  r18        ; Hierna is R18 = R18*2
4           lsl	 r18        ; Hierna is R18 = R18*2*2 (= R18*4)
5           pop	 r19        ; R19 laden met oude waarde R18 (let op pushes!)
6           add	 r18,r19    ; Hierna is R18 = R18*2*2 + R18 (= R18*5)
7           lsl	 r18        ; Hierna is R18 = (R18*2*2 + R18)*2 (= R18*10)
8           pop	 r19        ; Nu nog even R19 ophalen (eerste push)
9           ret             ; En terugkeren
\end{verbatim}

Zoals je ziet vermenigvuldigt deze routine R18 met 10. De truc is om gebruik
te maken van een combinatie van schuiven (vermenigvuldigen met 2) en optellen.
Eerst wordt R19 gepushd (regel 1) op stack omdat we R19 nodig hebben voor de
optelling (regel 5 en 6). Dan wordt R18 nog eens gepushd (regel 2) omdat later
die waarde nodig is voor de optelling via R19 (regels 5 en 6). Nu wordt R18
twee keer geschoven wat een vermenigvuldiging met 4 betekent (regels 3 en 4).
Dan moet er \'{e}\'{e}n keer de originele waarde van R18 bij worden opgeteld
om een vermenigvuldiging met 5 te krijgen. R18 is natuurlijk allang veranderd
dus we moeten een tweede register gebruiken. Dit is R19. R19 popt nu de stack
(en daar werd als laatste de originele waarde van R18 opgezet! Zie sheets week
4, pagina 25) en telt dit op bij R18 (regels 5 en 7). Dan wordt er nog
\'{e}\'{e}n keer geschoven en heb je als resultaat dat R18 vermenigvuldigd is
met 10!

Bij de code van opgave 19 worden weer drie uitspraken gegeven:

I. De instructie \texttt{lsl} schuift de inhoud van een register naar rechts 
II. Na het uitvoeren van \texttt{multi} is R18 vermenigvuldigd met 10 
III. R19 heeft na het uitvoeren van de eerste \texttt{pop}-instructie de waarde 0x{f}{f} 

I is niet waar: \texttt{lsl} schuift de inhoud naar links. Zie boek, blz 51.

II is waar: zie het verhaal hierboven.

III is niet waar: laatste push was die van R18 en de eerste pop is die van R19,
m.a.w. de (originele) waarde van R18 komt in R19 (zie sheets week 4, pagina 25).

Het antwoord is~\ref{ans:opg19}.

\vspace{1em}
Opgave \ref{opg:opg20}. Er worden twee uitspraken gedaan:

I is juist: je kan met \texttt{in r16,SREG} de vlaggen laden in register 16.
Zie sheets week 5, pagina 10.

II is ook juist: DDRC is dan wel een I/O-register maar die zijn gemapd in het
SRAM-bereik vanaf adres 0x0020. Zie sheets week 2, pagina 31. Je moet dat wel
het nieuwe adres even uitrekenen, en het is niet effici\"{e}nt: de
\texttt{lds}-instructie is twee words en de \texttt{in}-instructie is
\'{e}\'{e}n word.

Het antwoord is~\ref{ans:opg20}.

\vspace{1em}
Opgave \ref{opg:opg21}. Als je de instructie opschrijft die moet worden uitgevoerd
krijg je:

\begin{verbatim}
   R13 = R3 and R15

   acode = 0100
   muxrsr = 00011
   muxrsl = 01111
   endr = 1
   bcode = 000
   paddr = don't care
\end{verbatim}

Je hoeft alleen naar de bcode te kijken (in dit geval): alleen code 000 betekent niet springen, de overige codes betekenen wel springen, eventueel op de flag-conditie. Het antwoord is~\ref{ans:opg21}.

\vspace{1em}
\label{opg:ant22}
\ref{opg:opg22}. Hier zie je mooi een truc om het terugkeeradres aan te passen.
De truc is om in de subroutine twee keer een \texttt{pop} te doen. Hierdoor wordt het
gestackte terugkeeradres ''van de stack'' gehaald (in feite wordt alleen de stack
pointer verhoogd). Dan pushen we zelf twee waarden op de stack en voeren
vervolgens een \texttt{ret}-instructie uit!

\begin{verbatim}
rettrick: pop r16        ; pop stack, high byte RET-address
          pop r16        ; pop stack, low byte RET-address
          ldi r16,0x01   ; put 0x01 on stack as low byte for RET
          push r16 
          ldi r16,0x05   ; put 0x05 on stack as high byte for RET
          push r16 
          ret            ; And let's return!!!!!
\end{verbatim}

\newpage
Hieronder de stack tijdens uitvoering:

Stack direct n\'{a} uitvoering \texttt{call}: \hspace{20pt} Stack vlak
v\'{o}\'{o}r de uitvoering van \texttt{ret}:

\begin{verbatim}
    +------+                      +------+
    |  ..  |    <- sp             |  ..  | <-- sp           0x85d
    +------+                      +------+
    | 0x00 |                      | 0x05 |                  0x85e
    +------+                      +------+
    | 0x05 |                      | 0x01 |                  0x85f
    +------+                      +------+
\end{verbatim}

Het terugkeeradres is dus 0x0501. Het antwoord is~\ref{ans:opg22}.

\vspace{1em}
\ref{opg:opg23}. Deze lijkt wat lastig, maar je komt een heel eind als je
bedenkt dat de meeste instructies slechts \'{e}\'{e}n word beslaan. Alleen
\texttt{lds}, \texttt{sts}, \texttt{call} en \texttt{jmp}-instructies
beslaan twee words. Je kan \'{e}\'{e}n en ander opzoeken in de sheets van
week 3, pagina 2 t/m 15.

Het programma begint op adres 0x000 (zie de \texttt{.org}-directive),
\texttt{ldi} en \texttt{out} kosten ieder \'{e}\'{e}n word. De call kost twee
words en de \texttt{rjmp} (relative jump) kost \'{e}\'{e}n word. Alles tezamen
is dit 8 words en het adres subroutine \texttt{rettrick} is dus 0x007 want het
eerste word ligt op adres 0x000. Vergelijk het met array's in C. Een array van
4 items loopt van 0 t/m 3.

Het antwoord is~\ref{ans:opg23}.

\vspace{1em}
\ref{opg:opg24}. Zie het plaatje van de stack bij het antwoord van vraag
\ref{opg:opg22}.
Het laagste adres is 0x085d.

Het antwoord is~\ref{ans:opg24}.


\bigskip\bigskip
De antwoorden op een rij:
\medskip

\begin{tabular}{p{5cm} p{5cm} p{5cm}}
  \ref{opg:opg1} \ref{ans:opg1} & \ref{opg:opg9} \ref{ans:opg9}   & \ref{opg:opg17} \ref{ans:opg17} \\ 
  \ref{opg:opg2} \ref{ans:opg2} & \ref{opg:opg10} \ref{ans:opg10} & \ref{opg:opg18} \ref{ans:opg18} \\ 
  \ref{opg:opg3} \ref{ans:opg3} & \ref{opg:opg11} \ref{ans:opg11} & \ref{opg:opg19} \ref{ans:opg19} \\ 
  \ref{opg:opg4} \ref{ans:opg4} & \ref{opg:opg12} \ref{ans:opg12} & \ref{opg:opg20} \ref{ans:opg20} \\ 
                                &                                 &                                 \\ 
  \ref{opg:opg5} \ref{ans:opg5} & \ref{opg:opg13} \ref{ans:opg13} & \ref{opg:opg21} \ref{ans:opg21} \\ 
  \ref{opg:opg6} \ref{ans:opg6} & \ref{opg:opg14} \ref{ans:opg14} & \ref{opg:opg22} \ref{ans:opg22} \\ 
  \ref{opg:opg7} \ref{ans:opg7} & \ref{opg:opg15} \ref{ans:opg15} & \ref{opg:opg23} \ref{ans:opg23} \\ 
  \ref{opg:opg8} \ref{ans:opg8} & \ref{opg:opg16} \ref{ans:opg16} & \ref{opg:opg24} \ref{ans:opg24} \\ 
\end{tabular}   


%% Write all correct answers to file
\makeatletter
\AtEndDocument{%
  \if@filesw
	\immediate\openout\tempfile=answers.ans
	\immediate\write\tempfile
		{\@percentchar}
	\immediate\write\tempfile
		{\@percentchar\space Automaticly generated mastercard components}
	\immediate\write\tempfile
		{\@percentchar\space for exam \tisdexam@toetscode\space dd \tisdexam@toetsdatumkort}
	\immediate\write\tempfile
		{\@percentchar\space File written at \number\year/\two@digits\month/\two@digits\day} % doesn't behave well
	\immediate\write\tempfile
		{\@percentchar}
	\immediate\write\tempfile
		{}
	\immediate\write\tempfile
   		{\string\newcommand{\string\examcreatorname}{\tisdexam@opsteller}}%
	\immediate\write\tempfile
   		{\string\newcommand{\string\examstudentnumber}{00000000}}%
	\immediate\write\tempfile
   		{\string\newcommand{\string\examusername}{Mastercard}}%
	\immediate\write\tempfile
   		{\string\newcommand{\string\examcoursename}{\tisdexam@toetscode}}%
	\immediate\write\tempfile
   		{\string\newcommand{\string\examversion}{A}}%
	\immediate\write\tempfile
   		{\string\newcommand{\string\examdate}{\tisdexam@toetsdatumkort}}%
	\immediate\write\tempfile
   		{\string\newcommand{\string\examcomment}{Cesuur: 0 goed = 1, \nummultquestions\space goed = 10\string\newline\space Lineaire verdeling}}%
	\immediate\write\tempfile
   		{\string\newcommand{\string\examaq}{%
   			\getrefnumber{opg:opg1}\getrefnumber{ans:opg1},%
   			\getrefnumber{opg:opg2}\getrefnumber{ans:opg2},%
   			\getrefnumber{opg:opg3}\getrefnumber{ans:opg3},%
   			\getrefnumber{opg:opg4}\getrefnumber{ans:opg4},%
   			\getrefnumber{opg:opg5}\getrefnumber{ans:opg5},%
   			\getrefnumber{opg:opg6}\getrefnumber{ans:opg6},%
   			\getrefnumber{opg:opg7}\getrefnumber{ans:opg7},%
   			\getrefnumber{opg:opg8}\getrefnumber{ans:opg8},%
   			\getrefnumber{opg:opg9}\getrefnumber{ans:opg9},%
   			\getrefnumber{opg:opg10}\getrefnumber{ans:opg10},%
   			\getrefnumber{opg:opg11}\getrefnumber{ans:opg11},%
   			\getrefnumber{opg:opg12}\getrefnumber{ans:opg12},%
   			\getrefnumber{opg:opg13}\getrefnumber{ans:opg13},%
   			\getrefnumber{opg:opg14}\getrefnumber{ans:opg14},%
   			\getrefnumber{opg:opg15}\getrefnumber{ans:opg15},%
   			\getrefnumber{opg:opg16}\getrefnumber{ans:opg16},%
   			\getrefnumber{opg:opg17}\getrefnumber{ans:opg17},%
   			\getrefnumber{opg:opg18}\getrefnumber{ans:opg18},%
   			\getrefnumber{opg:opg19}\getrefnumber{ans:opg19},%
   			\getrefnumber{opg:opg20}\getrefnumber{ans:opg20},%
   			\getrefnumber{opg:opg21}\getrefnumber{ans:opg21},%
   			\getrefnumber{opg:opg22}\getrefnumber{ans:opg22},%
   			\getrefnumber{opg:opg23}\getrefnumber{ans:opg23},%
   			\getrefnumber{opg:opg24}\getrefnumber{ans:opg24}%
   		}}%
	\immediate\write\tempfile
   		{\string\newcommand{\string\examenglish}{no}}%
    \immediate\closeout\tempfile
  \fi
}% AtEndDocument
\makeatother

\fi

\end{questions}

\end{document}