%%
%% Example of an English exam for the faculty TIS of the The Hague University of Applied Sciences
%%

\documentclass[a4paper,english,addpoints,mimicwordtwentyten]{tisdexam}

\locatie{Delft}
\opleiding{Electrical Engineering}
\toetsnaam{MATH}
\toetsnaamkort{MATH}
\opsteller{J. op den Brouw}
\tweedelezer{B. Kuiper}
\groep{EP21, EP22, EQ2D}
\toetsdatum{Thursday 6 july 2017}
\toetsdatumkort{06-07-2017}
\tijd{13:00 h -- 14:30 h}
%\aantalbladzijden{4}
%\aantalvragen{1}
\opmerkingen{Please hand in your copy of the exam.}
\cesuur{Total grading points is \numpoints.}
\cursuscode{E-MATH-th1}
%\overig{}
%\bijlagen{}
%\formulebladen{}
%\eigenaantekeningen{}
\boekendictaten{Reader MATH}

\gelinieerdpapiertrue
%\opgavenbladentrue
%\ruitjespapiertrue
\kladpapiertrue
%\antwoordformulierabcdetrue
%\omslaggemaakttentamentrue
%\antwoordformulierjaneetrue
%\overigtrue
%\antwoordformulierjaneevraagtekentrue
%\bijlagentrue

\eenvoudigerekenmachinetrue
\tekenbenodigdhedentrue
\grafischerekenmachinetrue
\eigenaantekeningentrue
%\computertrue
\boekendictatentrue
%\formulebladentrue

\documentengesorteerddocumenttrue
%\documentengesorteerdstudenttrue

\usepackage{mathptmx}% http://ctan.org/pkg/mathptmx
\usepackage{mathtools}

\usepackage{parskip}

\begin{document}

% Create the cover page
\makecoverpage

% Set new page geometry
\newgeometry{bindingoffset=0.2in,left=1in,right=1in,top=1in,bottom=1in,footskip=.40in}

% Begin with the questions
\begin{questions}

\question[10]
Calculate $1+1$

\question
Please find de antiderivative of the following functions:
\begin{parts}
\part[10] $\displaystyle\int x\ln x\ \text{d}\,x$ {} \droppoints
\part[10] $\displaystyle\int \sin^2 x\ \text{d}\,x$ {} \droppoints
\part[10] $\displaystyle\int x^2\ \text{d}\,x$ {} \droppoints
\end{parts}

\question[20]
Given the function: $\displaystyle f(x) = x^3 + 3x^2 + 5x + 1$. Find the extremae.

\vspace*{1cm}

\hrulefill

\begin{figure}[!ht]
\centering
Grading table\vskip10pt

\gradetable[h]
\end{figure}

\end{questions}


\end{document}